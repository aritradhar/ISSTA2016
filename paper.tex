\documentclass{sig-alternate-05-2015}

%% =============================================================
% \def\tool{\textsc{Tool}\xspace}
% \def\papertitle{\tool: Memory-Efficient and Time-Bounded Runtime Monitoring for 
% Resource-Constrained Systems}
\def\papertitle{Efficient Context-Aware Runtime Monitoring for Finite State Properties}
\def\pdfauthors{}
\def\paperkeywords{}
%% =============================================================

%------------------------------------------------------------------------------
%                                Pre-package misc.
%------------------------------------------------------------------------------

\usepackage{color}

\interfootnotelinepenalty=10000
\sloppy

% You can tweak clickable link colors here:
\definecolor{linkcol}{rgb}{0,0,1}
\definecolor{citecol}{rgb}{0,0.5,0}
\definecolor{urlcol}{rgb}{0.3,0,0}

% Make pdflatex use letter size --md
\setlength{\pdfpagewidth}{8.5in}
\setlength{\pdfpageheight}{11in}

%------------------------------------------------------------------------------
%                                Use packages.
%------------------------------------------------------------------------------

\usepackage{tikz}
\usepackage{amsmath}
\usepackage{ragged2e}
\usepackage{cite}
\usepackage{txfonts}
\usepackage{fancyhdr}
\usepackage{amssymb}
\usepackage{fancyvrb}
\usepackage{graphicx}
\usepackage{times}
\usepackage{pifont}
\usepackage{xspace}
\usepackage{epstopdf}
%\usepackage{ctable}
%\usepackage{xcolor, colortbl}
%%\usepackage[belowskip=-10pt,aboveskip=5pt,small,labelfont=bf]{caption}
\usepackage[aboveskip=5pt,small,labelfont=bf]{caption}
\usepackage{subcaption}
% \usepackage[hyphens]{url}
\usepackage{url}

\newtheorem{theorem}{Theorem}

\DeclareCaptionType{copyrightbox}

\usepackage[bookmarks=true,%
bookmarksnumbered=true,%
colorlinks=true,%
linkcolor=linkcol,%
citecolor=citecol,%
urlcolor=urlcol,%
hypertexnames=true,%
pdfpagelabels]{hyperref}

% \usepackage{sty/algorithm2e}
\usepackage{sty/multirow}
% \usepackage{sty/flushend}
% \usepackage{sty/usenix}
\usepackage{sty/epsfig}
\usepackage{sty/endnotes}
\usepackage{sty/algorithm} 
\usepackage{sty/algorithmic}
\usepackage{listings}

\newcommand{\subparagraph}{}
\usepackage[small,compact]{sty/titlesec}

%------------------------------------------------------------------------------
%                                Space savers.
%------------------------------------------------------------------------------
%==========================================
%rename listing
\renewcommand\lstlistingname{Code}
%==========================================
% \SetAlFnt{\small}
% \SetAlCapFnt{\small}
% \SetAlCapNameFnt{\small}
%%\SetVlineSkip{0pt}

% \setlength\floatsep{5pt}
% \setlength\textfloatsep{5pt}
% \setlength\intextsep{5pt}

% Use a smaller font size for URLs:
\makeatletter
\def\url@myurlstyle{%
   \@ifundefined{selectfont}{\def\UrlFont{\small}}{\def\UrlFont{\small}}}
   \makeatother
\urlstyle{myurl}

%------------------------------------------------------------------------------
%                                Misc.
%------------------------------------------------------------------------------

% This adds ':' to the characters after which not to break URLs, and
% defines a smaller typewriter font. --cpk
% changed small to sf -- md
\def\UrlNoBreaks{\do:\do\(\do\[\do\{\do\<}%
\def\UrlFont{\small\ttfamily}
\def\UrlOrds{\do\*\do\~}%

% For referencing sections, Vern-style
\newcommand\xref[1]{\S~\ref{#1}}

\newcommand\fref[1]{Fig.~\ref{#1}}

% benchmarks
\def\bloat{\texttt{bloat}}
\def\pmd{\texttt{pmd}}
\def\chart{\texttt{chart}}
\def\avrora{\texttt{avrora}}
\def\hasnext{\texttt{HasNext}}
\def\unsafeiter{\texttt{FailSafeIter}}
\def\hashset{\texttt{HashSet}}
\def\dacapo{\textsc{DaCapo}}
\def\javamop{\textsc{JavaMop}}
\def\soot{\textsc{Soot}}
\def\java{\textsc{Java}}
\def\aspectj{\textsc{AspectJ}}

% Black filled circles with white number on it
% See Comprehensive LaTeX Symbol List --cpk
\def\blackI{\ding{182}}
\def\blackII{\ding{183}}
\def\blackIII{\ding{184}}
\def\blackIV{\ding{185}}
\def\blackV{\ding{186}}
\def\blackVI{\ding{187}}

\def\first{({\it i})\xspace }
\def\second{({\it ii})\xspace }
\def\third{({\it iii})\xspace }
\def\fourth{({\it iv})\xspace }
\def\fifth{({\it v})\xspace }

% Fine-tuning for table spacing. --cpk
\def\TblSpT{\rule[-1ex]{0pt}{0pt}}
\def\TblSpB{\rule{0pt}{2.5ex}}

% Squeezing out some space. --cpk
% http://www-h.eng.cam.ac.uk/help/tpl/textprocessing/squeeze.html
%
%\renewcommand\subfigtopskip{0pt}
%\renewcommand\subfigbottomskip{5pt}
%\renewcommand\subfigcapskip{0pt}
%\renewcommand\floatpagefraction{.9}
%\renewcommand\topfraction{.9}
%\renewcommand\bottomfraction{.9}
%\renewcommand\textfraction{.1}
%\setlength{\parskip}{0em}
%\frenchspacing

%------------------------------------------------------------------------------
%                                Space savers.
%------------------------------------------------------------------------------
% This mylist environment indents items, and saves less space than the above.
\newcounter{myctr}
\newenvironment{mylist}{\begin{list}{(\textbf{\arabic{myctr}})}
{\usecounter{myctr}
\setlength{\topsep}{1mm}\setlength{\itemsep}{0.5mm}
\setlength{\parsep}{0.5mm}
\setlength{\itemindent}{0mm}\setlength{\partopsep}{0mm}
\setlength{\labelwidth}{-2mm}
\setlength{\leftmargin}{0mm}}}{\end{list}}


\newcounter{myctr1}
\newenvironment{challenges}{\begin{list}{\textbf{C\arabic{myctr1}.}}
{\usecounter{myctr1}
\setlength{\topsep}{1mm}\setlength{\itemsep}{0.5mm}
\setlength{\parsep}{0.5mm}
\setlength{\itemindent}{0mm}\setlength{\partopsep}{0mm}
\setlength{\labelwidth}{-2mm}
\setlength{\leftmargin}{0mm}}}{\end{list}}

% Space saving List environment for itemizing.
\newenvironment{mybullet}{\begin{list}{$\bullet$}
{\setlength{\topsep}{1mm}\setlength{\itemsep}{0.5mm}
\setlength{\parsep}{0.5mm}
\setlength{\itemindent}{0mm}\setlength{\partopsep}{0mm}
\setlength{\labelwidth}{-2mm}
\setlength{\leftmargin}{0mm}}}{\end{list}}

\newcommand{\myparagraph}[1]{\noindent{\scshape \bfseries #1.}}

%------------------------------------------------------------------------------
%                               Fancy header setup.
%------------------------------------------------------------------------------
%
\pagestyle{fancyplain}
\lhead{}
\lfoot{}
\chead{}
\rhead{}
\cfoot{\thepage}
%%\rfoot{{\scriptsize \today}}
\renewcommand{\headrulewidth}{0pt}

%\setlength{\intextsep}{10pt plus 2pt minus 2pt}

%% Reducing margins further -- md
%%\addtolength{\hoffset}{-0.25in}
%%\addtolength{\textwidth}{0.5in}
%%\addtolength{\voffset}{-0.25in}
%%\addtolength{\textheight}{0.4in}

%------------------------------------------------------------------------------
%                                New commands
%------------------------------------------------------------------------------

% For notes to authors:
\newcommand{\note}[1]{{\textcolor{red}{[\textit{\textbf{TODO: #1}}]}}}
\newcommand{\code}[1]{\texttt{\small{#1}}}
% \newcommand{\code}[1]{\texttt{\scriptsize{#1}}}
\newcommand{\mycomment}[1]{}
\newcommand{\ignore}[1]{}
\newcommand{\todo}[1]{\textbf{TODO: #1}}
\newcommand{\mytab}{~~~~}
\newcommand{\sspace}{~}
\newcommand{\etal}{\textit{et al.}}
\newcommand{\eg}{{e.g.,}}
\newcommand{\ie}{{i.e.,}}
\newcommand{\lno}[1]{{\tiny{\textbf{(#1)~~}}}}

\newcommand{\docker}{Docker}
\newcommand{\tikzcircle}[3][black,fill=white]{\tikz[baseline=-0.75ex]\draw[#1,
radius=#2] (0,0) circle node[text=black] {\scriptsize #3};}%
%%\newcommand*\circled[1]{\tikz[baseline=(char.base)]{\node[shape=circle,draw,
%%inner sep=2pt] (char) {#1};}}
\newcommand{\rulesep}{\unskip\ \vrule\ }

% \SetEndCharOfAlgoLine{}

\renewcommand{\thetable}{\arabic{table}}

\newcommand*{\refname}{Bibliography}

\def\yes{\ding{51}}
\def\no{\ding{55}}

% The space-saving sledgehammer. -- md
%\renewcommand{\baselinestretch}{0.95}

\hypersetup{
pdfauthor = {},
pdftitle = {\papertitle},
pdfkeywords = {\paperkeywords},
pdfcreator = {LaTeX with hyperref package},
pdfproducer = {pdflatex}}

%lst
\definecolor{dkgreen}{rgb}{0,0.6,0}
\definecolor{gray}{rgb}{0.5,0.5,0.5}
\definecolor{mauve}{rgb}{0.58,0,0.82}

\lstset{%frame=tb,
  captionpos=b,
  language=Java,
  aboveskip=10pt,
  belowskip=0pt,
  abovecaptionskip=5pt,
  belowcaptionskip=5pt,
  numberblanklines=false,
  showstringspaces=false,
  columns=flexible,
  basicstyle={\scriptsize\ttfamily\bfseries\color{black}},
  numbers= left,
  numbersep=5pt,
  %frame=single,
  numberstyle=\tiny\color{black},
  keywordstyle=\bfseries\color{blue},
  commentstyle=\color{dkgreen},
  stringstyle=\color{mauve},
  breaklines=true,
  breakatwhitespace=true
  tabsize=2,
  xleftmargin=2em,
  morecomment=[l]{//},
  escapeinside={<@}{@>},
}

%% only for the tech report
%%\makeatletter
%%\def\@copyrightspace{\relax}
%%\makeatother

\begin{document}

% Copyright
\setcopyright{acmcopyright}
%\setcopyright{acmlicensed}
%\setcopyright{rightsretained}
%\setcopyright{usgov}
%\setcopyright{usgovmixed}
%\setcopyright{cagov}
%\setcopyright{cagovmixed}

% DOI
\doi{10.475/123_4}

% ISBN
\isbn{123-4567-24-567/08/06}

%Conference
\conferenceinfo{ISSTA '16}{July 17--20, 2016, Saarbrucken, Germany}

\acmPrice{\$15.00}

% %
% % --- Author Metadata here ---
% \conferenceinfo{WOODSTOCK}{'97 El Paso, Texas USA}
% %\CopyrightYear{2007} % Allows default copyright year (20XX) to be over-ridden - 
% IF NEED BE.
% %\crdata{0-12345-67-8/90/01}  % Allows default copyright data 
% (0-89791-88-6/97/05) to be over-ridden - IF NEED BE.
% % --- End of Author Metadata ---

\title{\Large \bf \papertitle}

% % ISSTA'16 is double blind, so removed the author block

\maketitle

\begin{abstract}
%%
\small

% % Runtime monitoring is employed in practice to ensure that a program shows 
% % expected behavior during its execution. Past decade has seen a prominent rise in 
% % the number of novel runtime monitoring frameworks and tools due to the promise 
% % shown by monitoring techniques. Many of these tools have been used effectively 
% % to verify typestate properties that are associated with legal Application 
% % Programming Interface (API) usages. In spite of their effectiveness, the tools 
% % have been occasionally found to incur significant runtime overheads, which could 
% % be even larger than the program's own execution time, particularly when the 
% % properties are associated with multiple objects. Moreover, the overheads have 
% % also been found to be extremely variable even when handling the events of the 
% % same kind. Such undesirable overheads may restrict the application of monitoring 
% % only to test environments or even worse, make it infeasible. In addition to time 
% % overheads, the monitoring tools also consume significant memory to keep 
% % monitors. 

% % Occasionally this extra memory outweighs the program's own memory requirements. 
% % All of these limitations make runtime monitoring infeasible for real-time 
% % programs that are typically executed in resource-constrained environments, where 
% % functional as well as non-functional requirements are critical.In this work, we 
% % propose a monitoring framework that investigates the trade-off between the 
% % runtime overhead and the error reporting. Our approach is motivated by the fact 
% % that there is a large redundancy among monitors in terms of their behavior which 
% % results in many monitors detecting same errors. The approach works by limiting 
% % the number of monitors based on heuristics that are related to the program 
% % execution context. In addition, the approach also limits the number of monitors 
% % that are associated with events related to a set of objects. As a result, the 
% % framework enables monitoring which consumes much less memory and provides 
% % worst-case bounds for the execution times of handling events. Moreover, our 
% % study based on some challenging typestate properties and DaCapo benchmarks 
% % indicates that our approach does not result in extra memory overhead. At the 
% % same time, it detects all the violations that an un-optimized approach detects.

Runtime monitoring is employed in practice to ensure that a program shows 
expected behavior during its execution. However, monitoring has been 
occasionally found to incur large and unpredictable overheads in terms of memory 
and time, which makes its application challenging in resource-constrained 
environments. Past research has shown that this challenge is particularly big 
when the properties the programs are monitored for are finite state, 
parameterized, and associated with objects that are created in large numbers. In 
this work, we propose a monitoring approach that investigates various 
trade-offs associated with the overheads of monitoring, the execution times of 
monitoring operations, and the error reporting. Our approach is motivated by two key observations.
First, there is a prominent behavioral redundancy among monitors resulting in more 
than one monitors detecting same errors. Second, the events on the same or related objects are often
temporally separated from the rest. We leverage these 
observations to reduce the number of monitors, and provide compact worst-case 
time bounds to monitoring operations, without compromising much with the 
soundness of the optimized monitoring system. As a side-effect the overall time overhead also decreases.
Evaluation of the prototype 
implementation of our approach using challenging combinations of finite state properties and 
DaCapo benchmarks indicates that the approach can effectively control the 
overheads and worst-case execution times.

%%
\end{abstract}

% \input{sections/ccs}
\section{Introduction}
\label{sec:introduction}

Modern software applications are complex and diverse in nature. They come in 
various forms ranging from large web-based systems that serve large number of 
user requests in reasonable time to applications that run on small hand-held 
mobile devices. Irrespective of their forms, they pose challenges for tools that 
perform automatic verification either statically or dynamically. Even though 
useful, static analysis techniques often produce numerous false positives 
that are hard to analyze~\cite{Deline04,Naeem:ECOOP08}, or need
annotations from developers~\cite{Bierhoff:ECOOP09, Bierhoff:OOPSLA07} which puts
an extra burden on them. Hence, researchers have invested time and 
effort to develop runtime monitoring tools and techniques that are precise
and can scale to realistic programs \cite{Allan:OOPSLA05, Arnold:OOPSLA08, chen2005, Reger2015}.

In spite of their effectiveness, monitoring tools have been occasionally found 
to incur considerable memory and execution overheads making their deployment 
challenging even
% in production or even 
test environments ~\cite{Purandare:2013}.
% challenging. 
Monitoring can often consume memory and compute resources 
% that are even 
greater than those 
% the ones 
consumed by the monitored program itself,
% being monitored 
especially when properties of interest are complex and are associated with objects 
generated in large numbers, such as collections and iterators~\cite{chen2009}. 
Monitoring overhead can be a big concern considering that in practice 
programmers would like to track several properties at a time, which cumulatively 
adds the overheads of individual properties~\cite{luo-2014, Purandare:2013}.

The monitoring overheads are unpredictable since they depend on the program and 
property interactions. For a given program and property, the interactions depend 
on the executed program paths. To make matters worse, 
monitoring operations corresponding to an event may take arbitrarily long when 
the properties are associated with multiple objects.
These operations can consume variable 
execution times even for similar events owing to the fact that variable number 
of monitors may be associated with each one of those events at different times
and each one of the 
monitors needs to be tracked after the occurrence of these events. This number 
can grow rapidly and is theoretically unbounded. In one study, up to 1548 monitors
associated with a single object have been reported ~\cite{Purandare:2013}.

Large and unpredictable overheads pose serious challenges to monitoring since 
they adversely impact the system performance.
%For real-time embedded 
%applications, developers need to provide bounds for the worst case execution 
%times of system operations.
%As a result, the usage of runtime monitoring is 
%restricted either to testing environments or to the production environments in 
%which resources are abundant and performance requirements are less stringent. 
Even though resource-constrained systems such as mobile and real-time embedded 
systems have more restrictive resource requirements, all modern systems 
including
% the embedded as well as many
web-based and cloud-based systems have practical constraints on their resources 
considering that the applications they run are often CPU- and memory-intensive, 
and their performance is expected to be high and predictable. Hence, in order to 
employ runtime monitoring in production environment, we need novel and efficient 
techniques that consume less resources and provide guarantees about their 
performance. At the same time, the techniques should also be effective in 
catching property violations, which is the primary objective of runtime 
monitoring.

Researchers have tried several approaches in the past, and developed tools to 
monitor programs efficiently by keeping the overall overheads within limits by 
turning off monitoring if it exceeded the allocated time
budget~\cite{Arnold:OOPSLA08, BartocciGKSSZS12, StollerBSGHSZ11}. 
However, these approaches do not deal with the properties that are related to 
multiple objects, and do not give guarantees about the worst-case execution 
times of monitoring operations. Moreover, they do not directly control the 
memory overheads. Other approaches either do not deal with finite-state 
properties or not perform inline monitoring. Finite state property 
monitoring allows us to check programs for properties that are more expressive, 
whereas inline monitoring allows us to keep the detection latency within limits, 
and also provides opportunities for performing evasive actions avoiding 
failures~\cite{DwyerPP10}.

In this work, we propose a novel inline monitoring technique that investigates 
the trade-off between limited resources, namely memory and monitor execution 
times, and the error reporting. Our approach is motivated by a key 
observation that there is large redundancy among monitors in terms of their 
behavior that results in many monitors detecting the same error. However, 
programmers are interested in catching distinct errors, rather than several 
instances of the same error. Our technique works by limiting the number of 
monitors based on heuristics related to the program's execution context. When 
the properties are related to multiple objects, there can be several monitors 
waiting for an event. Since the number of these monitors is unbounded, the 
technique puts a hard limit on the number of monitors associated with the 
event. As a result, 
it consumes much less memory and provides tight worst-case bounds on the 
execution times of handling events.

The second key observation that we make is about the occurrence of events.
We observe that he events on the same or related objects often occur together
and are temporally separated from the rest. As a consequence, we observe
that newly created objects are likely to be associated with recent events.
We make use of these observations to develop heuristics to prioritize monitors
in the allocation to maintain the soundness of the system.

%Our study based on some challenging finite-state properties and DaCapo benchmarks 
%indicates that the technique has a potential to detect all distinct violations that an 
%un-optimized approach could detect. Moreover it can detect these violation using
%much less resources in terms of memory and execution time.

This paper makes two major contributions.
% of our research is double-fold. 
First, we present a novel approach~\xref{sec:approach} that is memory-efficient 
and time-deterministic. Second, we present a study of a prototype 
implementation~\xref{sec:implementation} of our approach on realistic 
benchmarks and properties. Our evaluation~\xref{sec:evaluation} based on some
challenging Java standard library properties and \dacapo\ benchmarks 
indicates that the technique has a potential to detect all distinct violations that an 
un-optimized approach could detect. Moreover it can detect these violation using
much less resources in terms of memory and execution time.


%indicates the effectiveness of our approach.

\ignore{Runtime monitoring is employed in practice to ensure that a program 
shows expected behavior during its execution. Past decade has seen a prominent 
rise in the number of novel runtime monitoring frameworks and tools due to the 
promise shown by monitoring techniques. Many of these tools have been used 
effectively to verify typestate properties that are associated with legal 
Application Programming Interface (API) usages. In spite of their effectiveness, 
the tools have been occasionally found to incur significant runtime overheads, 
which could be even larger than the program's own execution time, particularly 
when the properties are associated with multiple objects. Moreover, the 
overheads have also been found to be extremely variable even when handling the 
events of the same kind. Such undesirable overheads may restrict the application 
of monitoring only to test environments or even worse, make it infeasible. In 
addition to time overheads, the monitoring tools also consume significant memory 
to keep 
monitors. Occasionally this extra memory outweighs the program's own memory 
requirements. All of these limitations make runtime monitoring infeasible for 
real-time programs that are typically executed in resource-constrained 
environments, where functional as well as non-functional requirements are 
critical.In this work, we propose a monitoring framework that investigates the 
trade-off between the runtime overhead and the error reporting. Our approach is 
motivated by the fact that there is a large redundancy among monitors in terms 
of their behavior which results in many monitors detecting same errors. The 
approach works by limiting the number of monitors based on heuristics that are 
related to the program execution context. In addition, the approach also limits 
the number of monitors that are associated with events related to a set of 
objects. As a result, the framework enables monitoring which consumes much less 
memory and provides worst-case bounds for the execution times of handling 
events. Moreover, our 
study based on some challenging typestate properties and DaCapo benchmarks 
indicates that our approach does not result in extra memory overhead. At the 
same time, it detects all the violations that an un-optimized approach detects.
}

\ignore{
From batch jobs running on mainframes to applications running on PCs, we have 
all lived through the shifts in the size and complexity of the modern software 
systems. With this growth, the expectation about the performance and reliability 
has also grown at the same time. Often in modern programming languages, 
programmers developing large applications face the problem of obeying many 
restrictions in order to guarantee the correctness of the program. This may be 
done irrespective of the actual functional requirements that the program may 
have. Programming errors occur frequently during software development as abiding 
to all these programming rules is quite tough. Moreover, the software system may 
cause a failure only under unusual conditions that may be missed during testing. 
Thus, a need for better quality control of the software development process has 
given rise to the program analysis and often identifying and removing these 
errors can consume a large fraction of a piece of software's development 
cost.\\\\
Some of the programming restrictions can be enforced by the programming 
language's type system. Errors that cannot be detected by type checking or by 
conventional static scope rules are detected by typestate tracking. To ensure 
that a program behaves correctly, typestate analysis techniques are used. This 
approach lets programmers to check a large range of program properties, often 
called \textit{typestate properties}. An object is not isolated; it interacts 
with other objects. At any time, an object is in some state, and the state 
changes when an operation is performed on that object. A typestate [3] analysis 
models every possible state throughout its lifetime. Typestate analysis can be 
used to analyse whether a given program violates typestate property. One can 
identify each safety property with a set of ?Äúbad?Äù finite execution traces, 
with the intuition that once one of those states is reached, the safety property 
is violated. These rules can often be expressed as a regular expression and 
modelled as a 
finite state automaton (FSA). For instance, programmers must call 
getInputStream( ) after a preceding call to connect( ).

\begin{figure}[h]
  \centering
    \includegraphics{./images/Fsa.png}
  \caption[FSA]{Partial typestate specification for java.net.Socket.}
  \label{fig:FSA}
\end{figure}


Figure 1.1  shows  a  finite state automaton providing a partial specification 
for the   java.net.SocketAPI. The finite state machine expresses the language of 
all program executions that violate this property. It monitors a connection's 
'close()', 'connect()' and 'getInputStream(),getOutputStream()' events and 
signals an error at its accepting state.\\\\
There are many typestate checking tools, both dynamic [4,7,8,13] and 
static[1,2,5,6], developed to ensure the correctness of property. While static 
program analysis inspect the program?Äôs code to prove the absence of typestate 
property violations for all possible executions of a given program, dynamic 
analysis tools incorporates a runtime monitor in the program under test that 
realizes the typestate properties that the program must satisfy. Testing all 
possible execution paths of complex systems endures high costs. Inorder to 
improve the
cost-effectiveness of static typestate analyses, researchers have combined 
multiple techniques [17, 18]. In [17], the authors present an intraprocedural 
analysis that eliminates the need for expensive whole-program analysis. The 
researchers have make use method annotations in the form of access permissions 
that specify typestate changes. [18] present a tool \textit{Plural} based on 
this approach and evaluate it using a few applications. Although impressive in 
many ways, these approaches still may produce false positives. Also, abnormal 
behaviour can be caused due to the deployment configurations and usage scenarios 
which cannot be examined by statically based testing techniques. \\As a result, 
dynamic analysis or runtime monitoring has gained considerable attention over 
static analysis. Researchers in runtime verification have developed powerful 
runtime monitoring tools [4,7,8,9,13]. These tools instrument the program under 
test with a runtime monitor and then, by composing the monitor with the program, 
the 
monitor observes the occurrence of each transition and decides whether the 
properties have been met or violated. Events are generated as a result of the 
instrumentation which keeps track of typestates. One important drawback of 
monitoring is that the instrumentation added to the program under test can yield 
significant overhead which hinders the monitoring of the application in 
practice. Depending on the program and property that are monitored, the overhead 
can vary significantly because the overhead depends on both the number of 
monitors that are created and the number of events generated during the program 
execution [14].\\\\
There have been several attempts earlier to optimise the runtime monitoring. 
[15, 16] present optimization techniques to reduce the overhead caused by the 
runtime overhead. They present techniques to remove unnecessary monitor 
instances. A number of hybrid techniques which combine static analysis and 
dynamic analysis to reduce the overhead have been proposed. In recent years the 
researchers[10, 12] have tried to apply the hybrid approach in which typestate 
property violation is first checked by static analysis to reduce the number of 
monitors at runtime monitoring. In [7], Arnold et al presents QVM that checks 
violations of correctness properties and has an overhead manager that enforces 
an overhead within limits. It is built on JVM so, comes with the cost of non 
portability.\\\\ 
In our research, we aim to create a framework based on sampling of objects for 
runtime monitoring that is deterministic and memory efficient. In our approach, 
we investigate the trade-off between the runtime overhead and the error 
reporting.  We are using a novel approach that works by limiting the number of 
monitors based on heuristics that are related to the program execution context. 
Depending on whether an object that is analysed has been previously monitored or 
not, the decision to sample the objects is made. In other words, monitoring will 
be performed only for the sampled objects to ensure that there is  redundancy 
among monitor's behavior of detecting the same error again and again. The total 
number of monitors generated for analysing the violations is restricted. In 
addition, the approach also limits the number of monitors that are associated 
with events related to a set of objects. Only a limited number of monitors are 
generated, thus utilizing memory efficiently. We have presented a cost model 
that aims to provide worst-case bounds for the execution times of handling 
events.\\\\
We implemented the sampled object based runtime monitoring framework on JavaMOP 
[4]. We used our framework to monitor two typestate properties, HasNext and 
UnsafeIterator, on some of the benchmarks from the Dacapo benchmark suite[11] 
and were able to capture the violations even after limiting the generation of 
monitors. We have also defined a cost model that gives the cost incurred in 
runtime monitoring by our approach to show that it is deterministic in terms of 
execution time.\\\\
\textbf{Outline} The rest of the report is as follows: Chapter 2 explains the 
motivation behind this research work. Chapter 3 provides a detailed overview of 
Sampled Object based Monitoring and shows an example. Chapter 4 presents the 
Cost Model of our approach. Chapter 5 presents our experimental data. Chapter 6 
discusses the related work. Chapter 7 provides some concluding remarks.
}

\setlength\floatsep{5pt}
\setlength\textfloatsep{5pt}
\setlength\intextsep{5pt}

\section{Background and Motivation}
\label{sec:motivation}

\begin{figure}[t]
\centering
  \includegraphics[scale=0.3, trim=0 5cm 0 6cm]{./images/unsafeiterator.pdf}
  \caption[UnsafeIterator Property FSA]{UnsafeIterator Property.}
  \label{fig:unsafeiteratorfsa}
\end{figure}

Figure~\ref{fig:unsafeiteratorfsa} depicts a finite state automaton (FSA) that 
models a property \texttt{UnsafeIterator} which is defined by the API of Java 
standard library and is related to a \texttt{Collection} as well as an 
\texttt{Iterator} object. It specifies the rule that a collection should not be 
updated while it is being iterated. The symbols \textit{create}, 
\textit{update}, and \textit{next} correspond to creating an iterator from a 
collection, modifying a collection, and iterating over a collection 
respectively. Hence, as shown in the figure, the symbol \textit{next} observed 
after the symbol \textit{update} would indicate iteration after a collection 
update which pushes the FSA to the \textit{error} state. Monitoring systems 
typically instantiate a monitor corresponding to every such pair of a collection 
and an iterator and use it to track the FSA states depending on the sequence of 
operations or events encountered.

In practice, a collection object may get iterated several times in its life 
time. On the occurrence of a \textit{creation} event, that may correspond to the 
call to \textmd{Iterator()} method in the \textmd{Collection} interface that 
returns an iterator, a monitoring system may create a new monitor for tracking. 
The newly created monitor is associated with the collection object and also the 
iterator object. It is easy to see that since the collection and iterator are 
frequently used objects in many programs the number of monitors may grow 
quickly. This puts heavy burden on the system and the garbage collector. 
Moreover, if the monitor objects are live, garbage collector cannot reclaim 
them.

Similar issue arises with another iterator property \texttt{HasNext} depicted in 
Figure~\ref{fig:hasnextfsa}. It specifies that a call to the \textit{hasNext} 
method must be given before calling \textit{next} method to ensure that an object 
exists before it is used. Collections and iterators are not only commonly used 
objects, but often the sources of error \cite{}. 

\begin{figure}[t]
\centering
  \includegraphics[trim=20cm 0cm 15cm 1cm, scale=0.2]{./images/HasNext.pdf}
  \caption[HasNext Property FSA]{HasNext Property.}
  \label{fig:hasnextfsa}
\end{figure}

Monitoring finite state properties efficiently has been a challenge due to the 
large number of monitors that may get created and also due to the large number 
of monitors that may get associated with some events. 
Table~\ref{table:numofmonitors} shows the number of monitors we observed while 
monitoring some DaCapo benchmarks using JavaMOP 2.3 for \texttt{HasNext} and 
\texttt{UnsafeIterator} properties.

\ignore{
\begin{itemize}
\item {\bf HasNext}: Every call to method \texttt{next} associated with an 
iterator must be preceded by a call to method \texttt{hasNext}.
\item {\bf UnsafeIterator}: A collection should not be updated while it is being 
iterated.
\end{itemize}
}


\begin{table}[t]
\centering
\begin{tabular}{|c|c|c|c|c|}
\hline
\multirow{2}{*}{} & \multicolumn{2}{c|}{HasNext} & 
\multicolumn{2}{c|}{UnsafeIterator} \\
\cline{2-5} 
                  & monitors     & events        & monitors         & events     
      \\ \hline
avrora            & 0.8M      & 1.5M       & 0.9M         & 1.4M          \\ 
\hline
bloat             & 1.9M      & 162M    & 1.9M          & 82M         \\ \hline
pmd               & 8.6M      & 47M      & 1.9M          & 26M         \\ \hline
\end{tabular}
\caption{Number of monitors created and events generated for different programs 
and properties}
\end{table}
\label{table:numofmonitors}

Table~\ref{table:consumedmemory} presents the memory consumption of the 
benchmarks with and without monitoring when JavaMOP 2.3 was used for monitoring. 
It shows that the large number of monitors results in the consumption of 
significantly large amount of memory in comparison with the benchmarks' memory 
requirements. Moreover, in practice, programmers would like to monitor their 
programs for all of the interesting properties together, not separately. This 
only means that runtime monitoring would become extremely difficult or even 
infeasible when properties are monitored simultaneously. Purandare et al. 
present a technique that compacts several monitors into one and tracks programs 
for multiple properties simultaneously \cite{}. However, even though the compaction 
reduces the footprint of a monitored program to a certain extent, the footprint 
can still be significant and prohibitively high for 
resource-constrained systems. This is owing to the fact that the compaction 
reaches its limit quickly since 
it involves taking a finite state automata (FSA) product which may grow 
exponentially.

\begin{table}[t]
\centering
\begin{tabular}{|c|c|c|c|}
\hline
%\multirow{2}{*}{} & \multicolumn{2}{c|}{HasNext} & 
% \multicolumn{2}{c|}{UnsafeIterator} \\ \cline{2-5}
 {} & Original & HasNext & UnsafeIterator\\
 \hline
avrora            & 0.1G      & 0.4G(300\%)       & 0.7G(600\%)           \\ 
\hline
bloat             & 0.5G      & 1.1G(120\%)    & 1.3G(160\%)              \\ 
\hline
pmd               & 0.5G      & 0.6G(20\%)     & 1.0G(100\%)             \\ 
\hline
\end{tabular}
\caption{Memory consumption for different programs and properties with and 
without monitoring. The figures in the parentheses are the percentage 
overheads.}
\end{table}
\label{table:consumedmemory}

In addition to the overall high overheads, for the properties associated with 
multiple objects, such as \texttt{UnsafeIterator} the cost of handling 
individual events can also be very high. This  happens since theoretically unbounded number of 
monitors may get associated with the events depending on the program and 
property interactions. As a result, the cost of tracking one single event can become 
unpredictably large. In fact, it was observed while monitoring for 
\texttt{UnsafeIterator} property for the \textsf{DaCapo} benchmark 
\textit{bloat} that as many as 1548 monitors had got associated with a single 
\textsf{Collection} object. Each of these monitors needs to be tracked when an 
event is seen by the corresponding receiver object. As a result, monitoring 
actions become nondeterministic in terms of time. Not being able to provide
bounds to the monitoring execution can 
restrict the usage of monitoring to nonreal-time systems or systems where 
performance guarantees are not essential. 
However, most systems in practice would suffer due to degraded performance if 
the memory overheads and execution times of monitoring cannot be controlled.

%We also made some fine-grain observations regarding the redundancy in the 
%behavior of monitors.  For the same DaCapo benchmarks and properties, 
%Table~\ref{table:joinpoints} presents the number of \textit{observable} 
%statements that generate monitoring events. Even though monitors are generated 
%in millions, observable statements are only a few. This indicates that there is 
%a reason to believe that the \textit{programming contexts} under which these 
%statements are invoked are limited. 

For the same DaCapo benchmarks and properties, we made some fine-grain 
observations regarding the redundancy in the behavior of monitors. We observed 
that the 42 errors reported while monitoring \textit{bloat}  for property 
\texttt{HasNext} could be partitioned into two classes of errors such that all 
errors in a class had a similar \textit{execution context}. Similarly, we could 
divide 333 errors reported by \textit{pmd} into two classes. For this work, we 
associated an execution context with a method calling sequence
%a call stack and the program counter,
and the data was collected  by considering limited execution context for efficiency 
reasons. These observations indicate that many monitors catch redundant errors, 
which are not so useful to developers. Ideally, the reporting should only be 
performed for distinct errors which means catching only thee distinct errors for 
\textit{bloat} and two for \textit{pmd} would suffice. Hence, we believe that 
monitoring performed by sampling objects based on the program execution context 
could be effective and would require less resources. 

In this work, we present a runtime verification approach that tries to exploit 
the redundancy in the monitor behavior and retain only the ones that show 
distinct behavior. As a result, it improves on memory efficiency as well as execution time 
determinism. The approach has been described in detail in 
Section~\ref{sec:approach}.




















\ignore{
Some very particular typestate properties are typically desired to be satisfied 
by all programs. The traditional runtime-verification approach has been 
developed over the past decade to analyze large complex applications because of 
several desirable properties. For instance, because the monitor specifications 
can be very expressive as runtime monitoring reason about concrete program 
events and runtime objects, and thus avoid false warnings. Also, when a runtime 
monitor detects a property violation, it can respond to this violation in many 
different ways, which can be any code from information logging to runtime 
recovery. The programmer has the guarantee that the monitor will detect a 
violation if it exists.\\
\begin{table}[h]
\centering
\begin{tabular}{|c|c|c|c|c|}
\hline
\multirow{2}{*}{} & \multicolumn{2}{c|}{HasNext} & 
\multicolumn{2}{c|}{UnsafeIterator} \\ \cline{2-5} 
                  & monitors     & events        & monitors         & events     
      \\ \hline
avrora            & 805500       & 1507792       & 909011           & 1365801    
      \\ \hline
bloat             & 1906736      & 162227791     & 1876472          & 82032703   
      \\ \hline
pmd               & 8576900      & 47231567      & 1949816          & 25958426   
      \\ \hline
\end{tabular}
\caption{Number of monitors created and events generated for different programs 
and properties}
\end{table}

On the other hand, existing tools for runtime monitoring usually incur an 
unacceptable overhead. Large number of monitors created during runtime and the 
events generated by the program execution are responsible for the overhead. In 
our experiments, for Dacapo benchmarks bloat and pmd benchmarks create nearly 
the same number of monitors, in excess of a few million, for property 
UnSafeIterator for JavaMOP[4]. Table 2.1 shows the large number of monitors 
created and the events generated during runtime of several programs from Dacapo 
Benchmark suite.
\\\\With the motivation that there is a large redundancy among monitors in terms 
of their behavior which results in many monitors detecting same errors, we 
investigate the trade-off between the runtime overhead and the error reporting. 
We propose a runtime verification framework that is memory efficient and time 
deterministic and can be used for real time embedded systems. The framework has 
the following characteristics:
\begin{itemize}
	\item \textit{Object based sampling}: we introduce object based sampling 
to collect sampled objects on which monitoring will be performed. The heuristics 
applied for sampling are related to the program execution context. We achieve it 
by obtaining the call-stack contents.  obtain the method calls invoked on the 
current runtime object under analysis and compare whether the trace of method 
calls of that object has been previously seen during runtime. 
	\item \textit{Deterministic monitoring}: For multi-object properties 
such as UnsafeIterator property, an event such as update can get associated with 
several monitors. This number can grow uncontrollably and can be as large as 
1548 for an event [9]. It becomes difficult to keep track of every monitor for 
that event. Hence, the whole operation of handling events may become non 
deterministic as far as timing requirements are concerned. We consider a 
monitoring behaviour to be deterministic as we can calculate the time taken to 
perform monitoring by the cost model presented. We are controlling the monitors 
associated with each event to make tracking deterministic.
	\item \textit{Limited number of monitors to make memory efficient}: we 
consider to limit the total number of monitors generated while runtime 
verification and at the same time maintaining sufficient accuracy for detecting 
property violations.
\end{itemize}
Runtime Monitoring is complete but fundamentally unsound since it cannot see 
anything beyond the current execution. In our approach, we sacrifice on the 
soundness to achieve memory efficiency and determinism in terms of time.	
}											
									
\section{Problem Definition}
\label{sec:definition}

We consider typestate properties or typestate-like properties~\note{Need to 
agree on what term to use. typestate is used here for the first time.} involving 
multiple objects. These properties are modeled using deterministic DFA 
$\mathcal{A}$ represented by the tuple $(Q, \Sigma, q_0, \delta, F, err)$, where 
$Q$ is the set of states, $\Sigma$ is the set of symbols, $q_0 \in Q$ is the 
start state, $\delta : Q \times \Sigma \to Q$ is the state transition function, 
$F \subset Q$ is the set of accept states, and $err$ is the designated error 
state. Typically, $F \bigcup err = Q$ and $F \bigcap err = \emptyset$. 

Let $\phi$ be the program property to be monitored, and let $O$ be the set of 
objects associated with $\phi$.  An event $\eta$ in the program under execution 
is represented by a pair $(\beta, \sigma)$, where $\beta \in 2^O$ is a set of 
objects associated with the event and $\sigma \in \Sigma$.

We define a monitor $m \in M$ parameterized by a pair $(\alpha, cur)$, where 
$\alpha \in 2^O$ and $cur \in Q$ is the current state of the monitor. We define 
a mapping $\psi : 2^O \nrightarrow 2^M$ that after receiving an event $\eta = 
(\beta, \sigma)$ returns a set of monitors $\theta$ corresponding to $\beta$, 
where $\beta \subseteq \alpha$. Note that there is a unique monitor associated
with $\alpha$, but there may be more than one monitor associated with $\beta$ 
when $\beta \subset \alpha$. The current state of every monitor that belongs to
$\theta$ is updated according to the symbol associated with the event. Formally, 
$\forall m \in \theta: m.cur \leftarrow \delta(m.cur, \sigma)$. We report an 
error when $m.cur = err$. \note{Not clear what is $\alpha$, and whether $\beta 
\subset \alpha$ or $\beta \subseteq \alpha$.}

\myparagraph{Our approach} A traditional unoptimized approach would use $\psi$ 
for locating monitors, whereas our approach generates a mapping $\psi': 2^O 
\nrightarrow 2^M$ such that $\psi' \subset \psi$. In other words, the approach 
ignores an event $\eta = (\beta, \sigma)$ if $\psi'(\beta)$ is undefined, even 
if $\psi(\beta)$ was defined. In addition, it return $\theta' \subseteq \theta$ 
to limit the monitors that need to be updated for an event.

A key challenge is not to compromise with error reporting. Even with reduced 
number of monitors, our approach still tries to detect all distinct errors. Let 
$P$ be the set of error reports generated by an unoptimized monitoring system. 
Let $\Pi$ be the set of programming contexts under which $P$ was generated. Let 
$\pi \in \Pi$ be the programming context under which the report $\rho \in P$ was 
generated.

\begin{theorem}
$\mid P\mid \ge \mid \Pi \mid$.
\end{theorem}

\begin{proof}
An error report is produced under exactly one programming context. However, a 
programming context $\pi_1$ can generate more than one error reports $\rho_1, 
\ldots ,\rho_n$. This is because during the program execution, if an erroneous 
state is reached under a programming context $\pi_1$, then the same state might 
also be reached every time under the context $\pi_1$ and an error report 
$\rho_k$ will be generated, where $1 \le k \le n$. Hence, the number of 
programming contexts under which an erroneous state can be reached cannot be 
greater than the number of error reports.
\end{proof}

We partition the set $P$ into blocks $b_1, \ldots , b_n$ as follows --- $\rho_i 
\in b_k$ and $\rho_j \in b_k$ where  $1 \le k \le n$ iff $\exists x$ \note{what 
is x?} such that $\rho_i$ and $\rho_j$ were created under the context $\pi_k$. 
In other words, all error reports that fall under one block are created under 
same programming context and hence we treat them equivalent. A key aim of our 
approach is to generate at least one report for every block. \note{'at least' 
or 'exactly' -- not sure of the wording here, so removed 'exactly', which 
seemed more stringent.} Formally, the approach strives to achieve a minimal $P' 
\subset P$ such that $\forall i : P' \bigcap b_i \ne \emptyset$ where $1 \le i 
\le n$ and $P$ is partitioned into $b_1, \ldots, b_n$.

\ignore{
The properties that we consider in this work are either typestate properties or
typestate-like properties that may involve multiple objects. The properties
can be modelled using a deterministic DFA $\mathcal{A}$
that can be represented by a tuple $(Q, \Sigma, q_0, \delta, F, err)$, $Q$ is the set of states,
$\Sigma$ is the set of symbols, $q_0 \in Q$ is the start state, $\delta : Q \times \Sigma \to Q$ is the state transition
function, $F \subset Q$ is the set of accept states, and $err$ is the designated error state. Typically, 
$F \bigcup err = Q$ and $F \bigcap err = \emptyset$.

Let $\phi$ be the property a given program is being monitored for, and let
$O$ be the set of objects associated with $\phi$.  An event $\eta$ in the program
under execution is represented by a pair $(\beta, \sigma)$,
where $\beta \in 2^O$ is a set of objects associated with the event and $\sigma \in \Sigma$.

We define a monitor $m \in M$ parameterized by a pair a pair $(\alpha, cur)$ where $\alpha \in 2^O$ and
$cur \in Q$ is the current state of the monitor. We define a mapping $\psi : 2^O \nrightarrow 2^M$ which after receiving an event
$\eta = (\beta, \sigma)$ returns a set of monitors $\theta$
corresponding to $\beta$ where $\beta \subseteq \alpha$. Note that there is a unique monitor associated
with $\alpha$, but there may be more than one monitors associated with $\beta$ when $\beta \subset \alpha$.
The current state of every monitor that belongs to
$\theta$ is updated according to the symbol associated with the event. More formally,
$\forall m \in \theta: m.cur \leftarrow \delta(m.cur, \sigma)$. When $m.cur = err$, an error is reported.

A traditional unoptimized approach would use $\psi$ for locating monitors, whereas our proposed approach
generates a mapping $\psi': 2^O \nrightarrow 2^M$ such that $\psi' \subset \psi$.
In other words, the approach ignores an event $\eta = (\beta, \sigma)$ if $\psi'(\beta)$ is undefined, even if
$\psi(\beta)$ would have been defined. In addition, it return $\theta' \subseteq \theta$ to limit the monitors
that need to be updated for an event.

The key challenge in this case would be not to compromise too much with the error reporting. With the reduced
number of monitors our approach still tries to detect all distinct errors. Let $P$ be the set of error reports
generated by an unoptimized monitoring system. Let $\Pi$ be the set of programming contexts under which
$P$ was generated. $\pi \in \Pi$ be the programming context under which the report $\rho \in P$ was generated.

\begin{theorem}
$\mid P\mid \ge \mid \Pi \mid$.
\end{theorem}

\begin{proof}
An error report is produced under exactly one programming context. However, a
programming context $\pi_1$ can generate more than one error reports $\rho_1, \ldots ,\rho_n$. This is because during the program
execution, if an erroneous state is reached under a programming context $\pi_1$, the same state might be reached every time
under the context $\pi_1$ and an error report $\pi_k$ where $1 \le k \le n$ will be generated. Hence, the number of programming
contexts under which an erroneous state can be reached cannot be greater than the number of error reports.
\end{proof}

We partition the set $P$ into blocks $b_1, \ldots , b_n$ as follows. \\
$\rho_i \in b_k$ and $\rho_j \in b_k$ where  $1 \le k \le n$
iff $\exists x$ such that $\rho_i$ and $\rho_j$ were created under context $\pi_k$.
In other words, all error reports that fall under one block are created under same
programming context and hence we treat them equivalent.

The error reporting goal of our approach is to generate at least and ideally, exactly one report
for every block. More formally, the approach strives to achieve a minimal $P' \subset P$
such that $\forall i : P' \bigcap b_i \ne \emptyset$ where $1 \le i \le n$ and $P$ is partitioned into
$b_1, \ldots, b_n$.
}

\section{Memory-Efficient and Deterministic Monitoring}
\label{sec:approach}

Our monitoring approach strives to reduce the memory requirements of finite state monitoring. At the same time it also tries to make monitoring more deterministic. The approach is based on the following mechanisms:

\paragraph{Context-based sampling} We use context-based sampling to control the number of monitors. The motivation for this approach comes from our observation that the monitors created at the same creation site and under similar context tend to go through a similar life cycle. In other words, these monitors are more likely to show a redundant behavior. Hence, the heuristic applied for this sampling is based on the program execution context. It works by i) identifying the monitor creation sites which are specified as creation events in the monitoring specification, and then by ii) obtaining the call-stack to understand the execution context, and finally by iii) making a decision about the allocation of the monitor based on the number of times this context was seen in the past. More often the monitoring system has seen the context, less likely it is to allocate a monitor. The programming context that we consider is limited to the the top three methods in the call stack along with the program counter.

\paragraph{Fixed-size global pool of monitors} We limit the total number of monitors that our monitoring approach would generate by creating a small monitor-pool of fixed size prior to the program execution and then maintaining it during the execution. In short our approach reuse of monitors. The monitors after their usage can be returned back to the system. Moreover, if the pool runs out of monitors and the system needs a new monitor, a monitor currently being used is forcefully reclaimed and made available for the reallocation. The heuristic that we use to reclaim a monitor is based on the observation that the program events are often temporally separated. In other words, in an execution trace, the events related to a same receiver object are likely to be found together. We exploit this observation by making the last used potentially active monitor available for reallocation. This also means that if the object that is removed from tracking observes another event it will not be tracked, and in case the event had otherwise generated an error, the approach would now miss it. Potentially missing a few errors is a cost of our optimized monitoring approach, however, by using smart heuristics we can minimize \textit{false negatives}.

\paragraph{Fixed-size local pool of monitors for individual objects} For properties related to multiple objects such as \texttt{UnsafeIterator} property, an object may can get associated with several monitors in its life-time. In this case, an event that performs an operation on the object results in the state of every associated monitor getting updated. Hence, handling such events may become \textit{nondeterministic} as far as timing requirements are concerned. We consider a monitoring behaviour to be deterministic if we can calculate the worst-case time taken to handle every monitoring event. Binding the number of monitors associated with an object allows us to limit the worst case execution time for any monitoring operation associated with that object. Our approach implements this constraint by allocating a fixed-size local pool of monitors to individual objects.

Section~\ref{subsec:outline} gives the outline of our approach, and Section~\label{subsec:algorithm} presents the algorithm.

\subsection{Monitoring Approach}
\label{subsec:outline}

\begin{figure}[t]
\centering
  \includegraphics[scale=0.4, trim= 2cm 1cm 0 1cm]{./images/schematic.pdf}
  \caption[Schematic of Memory-efficient and Deterministic Monitoring System]{Schematic of Memory-efficient and Deterministic Monitoring System.}
  \label{fig:schematic}
\end{figure}

Figure~\ref{fig:schematic} shows the main elements of our approach. The program under execution generates events parameterized by objects which are accepted by our monitoring system, and each one of them is handed over to the monitor allocation component. This component based on the event type and some heuristics makes a decision about whether to allocate a monitor or not. For convenience, we specify the program points that generate creation events explicitly, so that the approach can leverage this information to make the decision. If the event is of creation type, the monitoring system checks the execution context and its tracking history to see is the context was already sen in the past, and if yes, then with that frequency. The system probabilistically skips the monitor allocation phase for frequently seen contexts. The contexts are maintained as trees in which a new context adding a new branch or branches and leaf node. Every leaf node uniquely defines a context and it keeps information about the number of times the context was seen. The system can be confi

In case the system chooses to allocate a monitor, it consults the global and the local monitor pools to see if a fresh monitor can be allocated or the the local  If allocation is required it consults the global pool as well as the local pool to see if a fresh monitor is available or an old one needs to be reused. This mechanism currently is implemented as a circular array which naturally preserves the chronological ordering. In case an old one is chosen for monitoring, that monitor is first removed from the existing local pools of the corresponding objects, and then reallocated to the new object under consideration.

The tracking of monitors for the non-creation events is similar to the conventional model. The only difference is that if no monitor exists in the maps corresponding to objects associated with an event, then monitoring is completely skipped for that event. It is assumed that the monitors corresponding to the associated objects were never created by the system. Skipping monitoring is a direct saving in terms of execution time.




\begin{algorithm}[t]
                      % enter the algorithm environment
\caption[Algorithm]{Monitoring Algorithm. $\phi$ = ($Q$,$\Sigma$,$\delta$,$q_{0}$,$F$), $e=(l,b)$ where e is an event and $l \in L$ be a set of associated objects and $b \in \Sigma$}          % give the algorithm a caption
\label{alg1}                           % and a label for \ref{} commands later in the document
\begin{algorithmic}[1]                  
   %\STATE \textbf{let} \textit{O} be the set objects that receive events
   \STATE \textbf{let} $\Sigma_{c} \in \Sigma$ be the set of creation symbols
   \STATE \textbf{let} \textit{threshold} : $\mathcal{N} \to \mathcal{R}$ be a function generating a threshold value
   \STATE \textbf{let} \textit{A} be the global circular array of monitors
   \STATE \textbf{let} \textit{ObjsMons} : $O \to \textit{MS}$ be a function
   \STATE \textbf{let} \textit{MonObjs} : $M \to L$ be a function
   \STATE \textbf{let} \textit{ObjsSym} be a binary relation over L and $\Sigma$
   \STATE \textbf{let} $\psi \in \Psi$ be a finite sequence of method structures
   \STATE \textbf{let} $\eta$ be the data structure holding the program execution contexts
   %\STATE \textbf{let} \textit{getFreq} : A function that takes $\psi \in \Psi$ and $\eta$  and returns \textit{ct} which is the number of times the context was seen
   %\STATE \textbf{let} \textit{incFreq} : A routine that takes $\psi \in \Psi$ and $\eta$  and $n \in \mathcal{N}$ and sets $\psi.count$ 
   
    \IF{$b \in \Sigma_{c}$}
        \STATE $\psi$  $\leftarrow$ getExecutionContextInfo()
        \IF{isMatch($\psi$, $\eta$) = TRUE}
           \STATE $\theta$ $\leftarrow$ threshold($\psi$, $\eta$)
        \ELSE
        	   \STATE $\theta \leftarrow 1$
        \ENDIF
        \STATE updateExecutionContextInfo($\psi$, $\eta$)
        \IF{Rand() $\leq \theta$}  
            	\STATE $m$ $\leftarrow$ $A$.nextMonitor($o$)
        		\FOR{$l' \subseteq$ MonObjs($m$)} 
            	%\IF{$\exists \sigma \in \Sigma : (l', \sigma) \in ObjsSym $}
		\STATE ObjsMons($l'$) $\leftarrow  ObjsMons(l') / \{m\}$
            	%\ENDIF   
        		\ENDFOR
        		\FOR{$l' \subseteq l $} 
            		%\IF{$\exists \sigma \in \Sigma : (l', \sigma) \in ObjsSym $}
			\IF{ObjsMons($l'$).size() = MAX\_MON}
				 \STATE $m' \leftarrow$ ObjsMons($l'$).first();
			  	\FOR{$l'' \subseteq$ MonObjs($m'$)} 
					\STATE ObjsMons($l''$) $\leftarrow$  ObjsMons($l''$) / \{$m'$\}
        				\ENDFOR
			\ENDIF
                		 \STATE ObjsMons($l'$) $\leftarrow$  ObjsMons($l'$) $\cup$ \{$m$\}
            		%\ENDIF   
        		\ENDFOR
        \ENDIF
 \ENDIF
 \FOR{$m$ $\in$ ObjsMons($l$)}
     \STATE $m$.cur $\leftarrow$ $\delta$($m$.cur,$b$)
     \IF{$m$.cur = err}
        \STATE report$\hspace{5pt}\textbf{error}$
     \ENDIF
 \ENDFOR
 
\end{algorithmic}

\end{algorithm}
\label{algo:monitoring}



\subsection{Monitoring Algorithm}
\label{subsec:algo}

Algorithm~1 depicts the steps that implement our monitoring scheme. Lines 9--32 describe the operations that are performed when a creation event is encountered and a new monitor may need to be allocated. Line 10 checks the program execution context. If the execution context is already seen which is checked at  line 11, then a \textit{threshold} value is generated based on the number of times the context is seen in the past; else if the context is unseen, the threshold is assigned the highest possible value which is 1. In either case Line 16 updates the execution context history. In our implementation it involves either adding new branches in the context tree if the context was unseen or only incrementing the frequency count filed of the leaf node corresponding to the known context.

Lines 17--31 describe the steps when the threshold value is found to be large enough to justify allocation of monitor which is checked by the condition at line 17. As a result, a new monitor from the global circular array is allocated. Lines 19--21 describe the steps to reclaim the monitor in case it is previously assigned. This step ensures that all previous bindings are removed and the monitor is ready for the new assignment.

Lines 23--28 describe the steps that ensure that the local monitor pool limit is not reached. In case it is, the oldest monitor in the pool is reclaimed first by removing it from all the lists of associated object maps before the new monitor is added in the lists as shown by line 29.

Finally, as shown in lines 33--38, the relevant monitors are retrieved and their states are updated. In case any of the state is the \textit{error} state, then the error is reported. This final step is similar to the conventional monitoring, except that no monitor will be tracked if the system does not see any monitor allocated.


\subsubsection{Memory-Efficiency and Time-Determinism}
\label{subsubsec:efficiencyanddeterminism}

The algorithm preallocates monitors from a pool of constant size, and then if required, the monitors are reused. In our study we varied the pool size from 100 to 100k monitors. This results in reducing the number of required monitors considerably especially for the challenging program and property combinations in which millions of monitors get generated. As our study indicates, our approach may result in a dramatic saving in the memory requirements of a software.

It is easy to see that the algorithm has worst-case time bounds for all of its steps making the algorithm time-deterministic. Fetching current execution context in the form of a call stack is an expensive but still a constant time operation. The execution context tree has a bounded depth which we have limited to three in our prototype implementation. Hence performing read or write operations on it as in lines 11 and 16 are time-bound operations.

Depending on the map implementations, all the map operations such as the ones on lines 20, 26, and 29 are constant-time operations. The \textit{for} loops on lines 19, 22 and 25 iterate only a small finite number of times depending on the number of objects involved in the event which is typically either one or two. The \textit{for} loop on line 33 executes at most \texttt{MAX\_MON} times since that is the limit on the size of local monitor pools associated with objects.

\subsubsection{Soundness, Memory, and Determinism: Tradeoffs}
\label{subsubsec:tradeoff}

There is unfortunately a tradeoff between soundness and memory and we need to choose one at the cost of other. However, runtime monitoring is inherently unsound \cite{}. It can only report what it has seen. This means program errors may not be reported if the paths that encounter them are not executed. We stretch this limitation a little bit further to achieve substantial benefits in terms of memory savings. Our technique should enable developers to use runtime monitoring in the resource-constrained system settings where the previous techniques for finite-state monitoring might have been seen infeasible.

Similar tradeoff exists between soundness and determinism in terms of time, but we believe that binding worst-case execution times has its own rewards especially in the domain of real-time systems. Hence, our approach tries to use smart heuristics based on our observations and experience so as to reduce false negatives and improve the soundness of the system. As indicated by the study presented in Section~\ref{sec:study}, our approach can save considerable amount memory and can also make monitoring time-determinitic. Moreover, it does not compromise with its \textit{completeness} by ensuring that no false positives are produced.

\ignore{
\section{Simple Code and Property Example:}
While JavaMOP represents state of art runtime monitoring, our goal is to limit the number of monitors associated with a set of objects. It typically models sequencing properties as finite state automata (FSA) and check whether a program satisfies them during runtime.  JavaMOP uses object based monitoring in which a monitor for a FSA property is a relation between a set of related objects created during program execution and a state of the FSA.  The execution of sequence of program statements (under test), related to both property and the set of related objects, corresponds to the sequence of FSA transitions.\\
\begin{figure}[h]
\centering
  \includegraphics[scale=0.6]{./images/Unsafe.jpg}
  \caption[UnsafeIterator Property FSA]{UnSafeIterator Property.}
  \label{fig:typestateProperty FSA}
\end{figure}
\\Let us assume we are monitoring UnsafeIterator property that says not to modify an object of class Collection while iterating over this collection at the same time. In this case, it would be unclear whether the program should iterate over the original contents or the modified contents of the collection. While iterating over a collection, the runtime monitoring ensures that the program behaviour is well-defined. In other words, the monitor checks that there should not be a call to any method that is updating the collection after an iterator is created and before an element is accessed by a call to method next. The regular expression for the property is (create ; next* ; update*) and Fig 3.1 shows the finite state automata for UnsafeIterator property.\\\\
Below shows a code snippet for a test program. The relevant statements for the above property are calls to methods iterator(), add(), and next() on appropriate receiver objects. The relevant statements in the fragment of code being monitored are instrumented with extra code i.e., the OBSERVE... calls to perform monitoring.\\\\
When the call to method iterator() is encountered, a monitoring event create is generated. The instrumented code maintains a map, keyed by set of related objects that have been involved in previous events. The values corresponding to the keys are the set of monitors that are associated with the objects. On handling the create event, if it is found that no monitors are associated with the collection \textit{m1} and iterator \textit{itr}, our approach will list the history of method calls on the object \textit{itr} (bar(a1,a2), third(a1,a2), second(a1,a2), first(a1,a2), main() from our example). It will analyse if there exists same method calls of previously monitored objects and will sample the object for monitoring if match is not found and it will store the history of method calls for this object. However, on finding a match, the object may or may not be considered for monitoring depending upon how frequently it has been monitored previously.
\begin{center}
\textbf{A code snippet}
\begin{lstlisting}

void main(String[] args) {
	  ArrayList a1 = new ArrayList();
     al.add("C");
	  al.add("A");
	  al.add("B"); 
	  ArrayList a2 = new ArrayList ();
	  first(a1,a2);
	}
void first(ArrayList a1, ArrayList a2) { 
	  second(a1,a2); 
	} 
void second(ArrayList a1, ArrayList a2) {
	  third(a1,a2); 
	} 
void third(ArrayList a1, ArrayList a2) {
	  bar(a1,a2); 
	}
void bar(ArrayList m1,Arraylist m2) {
	  Iterator itr = m1.iterator();
		   OBSERVE.create(m1,itr);
	  m2.add(?��X?��);
		   OBSERVE.update(m2);
     while(itr.hasNext()) {
	    Object element = itr.next();
		   OBSERVE.next(itr);
	  }
	}
    
\end{lstlisting}
\end{center}


Whenever an object is sampled for monitoring, the total number of monitors analysing the property are checked. If the number is below the specified limit, a new monitor is created and references to it are associated with the new keys \textit{m1} and \textit{itr} and on the other hand, if the number exceeds the limit, the previously generated monitor is reassigned for the current objects by changing the references. Thus, a limited set of monitors is used for analysing the property violations as done in runtime monitoring. 
}














\ignore{
\begin{algorithm}[h]
                      % enter the algorithm environment
\caption[Algorithm]{Monitoring Algorithm. $\phi$ = (Q,$\Sigma$,$\delta$,$q_{0}$,F), e=(l,b) where e is an event and l $\in \Sigma$ and b $\in O$ be the set of receiver objects}          % give the algorithm a caption
\label{alg1}                           % and a label for \ref{} commands later in the document
\begin{algorithmic}[1]                  
   %\STATE \textbf{let} \textit{O} be the set objects that receive events
   \STATE \textbf{let} $\Sigma_{c} \in \Sigma$ be the set of creation symbols
   \STATE \textbf{let} \textit{prob} determines the generation of monitor (Initializing it to 1).
   \STATE \textbf{let} \textit{MA} be the array of monitors
   \STATE \textbf{let} \textit{ObjMonsMap} : \textit{O} $\to$ \textit{MS} be a map
   \STATE \textbf{let} \textit{ObjsSym} be a binary relation over L and $\Sigma$
   \STATE \textbf{let} \textit{SE} be a finite sequence of method frames
   
    \IF{$b \in \Sigma_{c}  \lor \textit{ObjMonsMap(o)} = null$}
        \STATE $SE  \leftarrow retrieveMethodCalls(o)$
        \FOR{$SE' \subseteq SE$} 
           \STATE $match  \leftarrow
           isMatched(SE')$
        \ENDFOR
        \IF{$(match)$}
           \STATE $prob \leftarrow P(countOfMatches)$
        \ENDIF   
        \IF{$ ((\exists(rand)\subset RNG : rand < prob))$}  
            \IF{$sizeOf (MA) < limit$} 
                \STATE $ m \leftarrow new monitor(o),MA \leftarrow m $
            \ELSE
                \STATE $m \leftarrow retrieve(MA[index]) $
            \ENDIF   
            \STATE $monitoringFlag \leftarrow  true$    
        \ELSE
            \STATE $monitoringFlag \leftarrow  false$ 
        \ENDIF
        \FOR{$l' \subseteq l $} 
            \IF{$\exists \sigma \in \Sigma : (l', \sigma) \in ObjsSym $}
                \IF{$(monitoringFlag) $} 
                   \STATE $ObjsMons(l') \leftarrow  ObjsMons(l') \cup \{m\}$
                \ENDIF
            \ENDIF   
        \ENDFOR
 \ENDIF
 \FOR{$m \in MA$}
     \STATE $m.cur \leftarrow \delta(m.cur,b)$
     \IF{$(m.cur = err)$}
        \STATE report$\hspace{5pt}\textbf{error}$
     \ENDIF
 \ENDFOR
\end{algorithmic}

\end{algorithm}
}

\ignore{Runtime Monitoring is complete but fundamentally unsound since it cannot see anything beyond the current execution. In our approach, we sacrifice on the soundness to achieve memory efficiency and determinism in terms of time.	

The aim of our research is to develop optimization techniques without compromising heavily with the error reporting. We have developed techniques that identify objects in a program for which new monitors will be generated only if they satisfy the required probability related to the program execution context. We are guided by the heuristics that considers the history of method calls that have been invoked on the current object being monitored over the course of object?��s lifetime, from allocation to collection, as the decision factor to determine if the object has been previously monitored.\\\\ 
In object based monitoring, for every object that invokes a property for monitoring, an instance of monitor is created and all subsequent calls on those objects generate events[14]. It may be the case that an object A following a particular sequence of method calls, reaches a method where the typestate property of the current object is monitored for violation and there is another object B which follows the same method call trace as A and reaches the same method. Here, the conventional monitoring approach will create again a new monitor for the second method invocation to check for property violation although the objects of this method have been previously monitored. Thus, in our approach, we are sampling the objects that are to be monitored. The new monitor instances are generated for sampled objects and the objects are sampled by listing the history of method calls invoked on the objects.\\
Our approach is built on dynamic typestate analysis and limits the number of monitors that are associated with events related to a set of objects. Moreover, the total number of monitors generated for the typestate property checking is bounded explicitly in our work. A monitor pool is maintained and once the number of monitors generated reaches the limit specified, the already generated monitors are reassigned for checking the violation of the properties. This helps in utilizing the memory efficiently and reduces memory overhead.\\\\
For a multi-object property, those that involve more than one object such as \textit{UnSafeIterator} property, one \textit{Collection} object has several monitors associated with it at a time. This means that a single method call on that object can result in several monitor
update operations. The number of associated monitors can grow uncontrollably and it becomes difficult to keep a track of every monitor for that event, hence the whole operation of handling events may become non-deterministic in terms of execution time. By limiting the number of monitors associated, the events become bounded and  hence, our approach makes monitoring deterministic in terms of time. We provide worst-case bounds for the execution times of handling events.}

%\section{Cost Model}
\label{sec:costmodel}

Monitoring tools perform several operations that include monitor creation, maintain their pool, tracking the associated monitors and executing a transition on them. Depending on the number of monitors created and the number of events incurred, all these operations incur a cost. The whole operation of handling events may become non-deterministic as far as timing requirements are concerned because the number of monitors associated with an object grow uncontrollably. Thus, it becomes difficult to predict the time taken to execute the monitoring operation . In this section we present a cost model that can give designer an idea about the overheard incurred by our framework for runtime monitoring. Our model is guided by the model defined in [12] and the basic definitions and symbols are adopted from the earlier study [12].
\section{Basic Definitions:}
Let $\phi$ = (Q,$\Sigma$,$\delta$,$q_{0}$,F) be the FSA for the property being monitored and $\pi$ the trace of events generated by a program execution related to $\phi$. The event, $e_{i}$, is a pair ($b_{i}$, $l_{i}$) where $b_{i}$ $\epsilon$ $\Sigma$, the set of FSA symbols and $l_{i}$ is set of objects associated with $e_{i}$. Let E be the set of events where E consists of two types of events, $E_{c}$ $\subseteq$ E, the set of monitor creation events which corresponds to an event that creates a monitor and $E_{c} \subseteq$ E, the set of binding events that bind new objects to the old ones that already have associated monitors. The binding events clone the existing monitors associated with old objects to create new monitors. The set of objects $l$ involved in a binding event is divided into two partitions $l_{b}$, set of old objects already associated with monitors and $l_{\overline b}$, set of objects of the binding event.\\\\
 L = \{{$\tau_{b}$ : b $\in$ $\Sigma$}\} where L is defined as the set of sets of types of objects that may be associated with all symbols and  ��b the set of types of objects that may be associated with the symbol b. For example, L= \{\{Iterator\},\{Collection\},\{Collection,Iterator\}\} for the property UnsafeIterator.\\
We considered the following points while developing the cost model:
\begin{itemize}
\item The monitors are created in creation event and a pool of monitors is maintained and then they may be reassigned. Once the total number of monitors reaches the limit, monitors can be retrieved from the pool of monitors and reassigned to another object after deleting the references of previous object(s). The total number of monitors used for monitoring is fixed.
\item In order to provide fast access to monitors, the monitoring tool support indexing scheme. For each subset of $\sum$?�� that has same object types, an index map is created and set of all such maps is denoted by {$\Psi_{\phi}$}
\item For method call matching, the history of method call is abstracted by getStackTrace() and a data structure is maintained to keep track of the sequence of these calls. In case a match is not found, monitoring of the associated object will be performed.
\end{itemize}
\section{Cost Models:}
Detailed cost models are presented in this sub section that predicts the cost of runtime monitoring. The cost of monitoring, $C_{m}$ for $\phi$ gives the additive cost of processing each event which may vary depending on whether it is a creation event or not.
\begin{center}
$C_{m}$ = $\displaystyle \sum_{i=1}^{|\pi|}f(\pi_{i})$
\end{center}
The distinct components of the cost of processing an event are identified as below:
\begin{enumerate}
\item $C_{C}$ : Cost of creating monitors.
\item $C_{ST}$ : Cost of abstracting method call history.
\item $C_{TM}$ : Cost of call trace matching.
\item $C_{A}$ : Cost of maintaining monitor array.
\item $C_{R}$ : Cost of accessing and manipulating index trees.
\item $C_{I}$ : Cost of inserting monitors into pools.
\item $C_{V}$ : Cost of traversing monitor pools.
\item $C_{T}$ : Cost of performing transitions.
\end{enumerate}
Thus, the cost of processing the $i^{th}$ event is given by
\begin{center}
f($\pi_{i}$) = $C_{C}(\pi_{i})$+$C_{ST}(\pi_{i})$+$C_{TM}(\pi_{i})$+$C_{A}(\pi_{i})$+$C_{R}(\pi_{i})$+$C_{I}(\pi_{i})$+$C_{V}(\pi_{i})$+$C_{T}(\pi_{i})$
\end{center}

The total number of events do not matter but the individual events are all bounded in our approach.

\subsection{$C_{C}$ : Creating Monitors}
This cost is incurred in performing step 18 of the algorithm.To begin monitoring of $\phi$ for a set of related objects, monitor creation events, $E_{c} \subseteq$ E occur. We have put a limit on the number of monitors generated for making monitoring memory efficient. For creation events, if the number of total monitors is less than the limit, a monitor must be created and then inserted into the appropriate indexing structure. $|l_{i}|$ * $c_{c_{m}}$ reflects the cost of creating a monitor component, $c_{c_{m}}$, for each of the objects in the event, $|l_{i}|$. As mentioned earlier, monitors will be created only if the total number of monitors is less than the limit. So, the number of monitors created in the creation event will always be less than or equal to limit specified, thus providing the worst case bound.

For binding events, the monitors that are already associated with $l_{b}$ are cloned and clones are associated with $l_{\overline b}$. P is defined as a function that takes a monitor and a prefix of the trace $\sigma_{i}$ as input and outputs the set of monitor components.
Thus, 
\begin{center}
$C_{C} = \begin{cases}
			\displaystyle \sum_{m\epsilon \theta (l_{i_{b}},\pi_{i})}^{|l_{i}|*c_{c_{m}}} {|P(m,\pi_{i}) \cup l_{i}| * c_{c_{m}}} & e_{i} \epsilon E_{c} \lor E_{b} \\
            0 & otherwise
         \end{cases}$
         
\end{center}

Note, the binding events are present in \textit{UnsafeMapIterator} property and not in \textit{HasNext} or \textit{UnsafeIterator} property.
\subsection{$C_{ST}$ : Retrieving method call history}
This cost is incurred in performing step 9 of the algorithm. Every object undergoes various method calls in its lifetime. Here, we need to track the methods invoked on the object. The extraction of the method calls history is the most important step in our approach because it helps in deciding if monitoring is required. The cost associated with this step is constant and depends on the underlying JVM. Let $\eta : l_{c} \to ST$ (where ST is the Stack Trace) be a function that inputs object associated with the create event and outputs the method call trace.\\
\begin{center}
$C_{ST}=|l_{c}| * c_{st}$ \hspace{16pt}  $e_{i} \in E_{c}$
\end{center}
Note, this cost is associated with creation event only.


\subsection{$C_{TM}$ : Method Call Trace Matching}
This cost is incurred by step 11 of the algorithm.The extracted history of method calls are stored in an appropriate data structure. For each object related to the creation event, this data structure needs to be accessed in order to check if there exists previously created monitors associated with some object having the same call history. The cost of traversing and matching a method call from the data structure be $c_{tm}$. If a method call is not present in the data structure, it needs to be added, thus incuring the cost, $c_{am}$. We define a function $\xi$ which takes a datastructure as input and outputs a cost of traversing a method call. Formally,
\begin{center}
$\xi(m) = \begin{cases}
			c_{tm} & method call$ present$\\
            c_{tm} + c_{am} & otherwise
          \end{cases}$  
\end{center}
Thus,
\begin{center}
$C_{TM}=|\eta(l_{c})| * \xi(m)$ \hspace{20pt}  $e_{i} \in E_{c}$ 
\end{center}

\subsection{$C_{A}$ : Maintaining Monitor Array}
This cost is incurred by step 17 - 20 of algorithm. While processing a creation event, we maintain an array of monitors that has a specified limit. Whenever monitoring is to be performed, a check on the size of array is done to ensure the limit. If the size is below the limit, a new monitor is inserted in the array; the cost of insertion is $c_{ia}$. However, the cost changes to $c_{ra}$ if the size exceeds the limit as the monitors will be reassigned.
\begin{center}
$C_{A} \leq \begin{cases}
			limit * c_{ia} & sizeOf(array) \le limit\\
            limit * c_{ra} & otherwise
          \end{cases}$
\end{center}

\subsection{$C_{R}$ : Index Trees}
Object based monitoring incurs the cost of retrieving object maps but the associated costs arise in different ways. Let $\psi \in \Psi_{\phi}$ be some index tree and $\nu$ be a map. $\Psi_{\phi}$ can be partitioned into $\Psi'_{\phi}$, set of index trees for all objects involved in creation event ($\rho(\phi)$) and $\Psi''_{\phi}$, be the additional index trees for the objects that are involved in binding events. These additional index trees provides access to the already existing monitors for binding objects.
\begin{center}
$\Psi_{\phi} = \begin{cases}
			 \Psi'_{\phi} & \rho(\phi)\\
             \Psi'_{\phi} \cup \Psi''_{\phi} & otherwise
          \end{cases}$  
\end{center}
We denote the number of maps accessed in processing an event by a function H that takes an event and set of index maps and outputs the number of maps. On the basis of program context, we sample the objects to be monitored. If a decision is taken in creation event to skip monitoring, this cost is not incurred in non-creation events. However, this cost is incurred in creation event. In the worst case, every subset of L may have corresponding index trees which can be at max $\displaystyle \sum_{k=1}^{|L|} 2^{k}$, thus putting bound on H. So, the number of object maps to be accessed given by \textit{$H(e_{i}, \Psi_{\phi})$} can not be larger than this bound.   Let $c_{r}$ be the cost of retrieving a value from such map, $c_{c_{p}}$ be the cost of creating a map and that of adding an entry to a map is $c_{a{p}}$. We define $\kappa$ a function that takes a map and outputs a cost of adding an entry to it, if required. 

\begin{center}
$C_{R} = \begin{cases}
			 H(e_{i},\Psi_{\phi}) * c_{r} + \displaystyle \sum_{\nu \in H(e_{i},\psi_{phi})} \kappa(\nu) & e_{i} \in E_{c} \cup E_{b} \\
             H(e_{i},\Psi_{\phi}) * c_{r} & otherwise
          \end{cases}$  
\end{center}

\subsection{$C_{I}$ : Inserting Monitors}
Everytime a new monitor is created while processing a creation event, it has to be inserted into each of the index maps, the cost of insertion is, $c_{a_{m}}$. However, the cost changes for a binding event where all the cloned monitors need to be inserted in the map. Function G would give us the number of index trees that will be accessed to retrieve the monitor pools in which the new monitor will be inserted. The total cost incured is given by
\begin{center}
$C_{I} = \begin{cases}
			 |\Psi_{\phi}| * c_{a_{m}} & e_{i} \in E_{c} \\
             |G(l_{i},\Psi_{\phi})| * c_{a_{m}} & e_{i} \in E_{b} \\
             0 & otherwise
          \end{cases}$  
\end{center}

\subsection{$C_{V}$ : Traversing Monitors}
There are no pre existing monitors related to the event while handling a creation event corresponding to a creation symbol and thus no traversal cost. For other events, $\theta$ is defined to be the number of monitors associated with a set of objects for the trace $\pi_{i}$. Each monitor is accessed individually with a cost of $c_{v}$. This cost is incurred only when the decision of monitoring is taken in creation event. Thus
\begin{center}
$C_{V} = \begin{cases}
			 0 & e_{i} \in E_{c} \\
             0 & \sim(monitoring) \\
             \theta(l_{i},\pi_{i}) * c_{v} & otherwise
          \end{cases}$  
\end{center}

\subsection{$C_{T}$ : Performing Transitions}
There are no transitions as the creation events are initialized to an appropriate state. For non creation events, each monitor for the event is accessed and performs a transition on the stored FSA state, which costs $c_{t}$. Also, there are no transitions if the monitoring is skipped. Thus
\begin{center}
$C_{T} = \begin{cases}
			 0 & e_{i} \in E_{c} \\
             0 & \sim(monitoring) \\
             \theta(l_{i},\pi_{i}) * c_{t} & otherwise
          \end{cases}$  
\end{center}





\section{Implemetation and Artifacts}
\label{sec:implementation}

We have developed a semi-automatic prototype to optimize the \texttt{Aspect}
generated by \textsc{JavaMop}.


%The aspects are then woven using ajc 1.7 compiler, and 
%the resulting Java program is executed on HPC server running Cent OS 6.5 and
% JVM 1.7.0. 

\subsection{\textsc{DaCapo} instrumentation} 
\label{subsec:dacapoInstr}

We have used \texttt{Soot}~\cite{soot} to
instrument \textsc{DaCapo} benchmarks. \code{getStackTrace()} method can be used to
fetch current stack trace but it introduces overhead. To mitigate this, we have
instrumented each of the methods of \textsc{DaCapo} benchmark with a static
\code{integer} field which is populated with an unique method id represented by
a $16$ bit \code{integer}. Code snippet~\ref{snippet:instrument} provides one
example method and it's instrumented counterpart.

\lstset{escapeinside={/*@}{@*/}, language=Java , caption=\bf Method trace
implementation., label=snippet:instrument} \begin{figure}[t]
\begin{lstlisting}
void foo(){		//original method
..   //some code
}
void foo(){		//instrumented method
static int methodID = 0x1a;
MethodTrace.insertMethodTrace(methodID);
..   //some code
}
\end{lstlisting}
\end{figure}

\subsection{Lightweight Method Trace}
\label{subsec:trace}

We have also simulate the stack trace by using a
circular array which contains these method id's. The circular array is
implemented by a $64$ bit \code{long} in the format $id_i|id_j|id_k$, where
$id_i$ is the method id corresponding to method $\code{f}_i$ and is a $16$ bit
integer. A pseudo code of the method stack is described in
Code~\ref{snippet:methodTrace}.

\lstset{escapeinside={/*@}{@*/}, language=Java , caption=\bf Method trace
implementation., label=snippet:methodTrace} \begin{figure}[t]
\begin{lstlisting}
long trace; //method stack trace
int counter = 0;
void insertMethodTrace(int methodID) {
 switch (counter) {
  case 0: id <<= 32; break;
  case 1: id <<= 16; break
  case 2: id <<= 0; break;
  default: break;
 }
 trace |= id;
 trace &= 0xffffffffffffL;
 counter = (counter + 1) % traceLength;
}
\end{lstlisting}
\end{figure}
The stack trace (\texttt{trace} in code~\ref{snippet:methodTrace}) is populated
by bit-wise \texttt{shift} and \texttt{OR} operation. This makes any query
operation over the stack-trace very light weight compared to traditional
\texttt{getStackTrace()} call.

\subsection{Aspect Generation and Optimizations}
\label{subsec:aspectGen}

Our implementation is semi-automatic and is based on the refinement of the 
aspects generated by JavaMOP 2.3. The reason for choosing an older version of 
JavaMOP was that it is simple to understand and it performs less optimizations. 
Hence, the chance of interfering the existing optimizations with our approach is 
less. This would not have been possible if we did not have a good understanding 
of the aspects consumed by our prototype implementation. Moreover, we compare 
the results of our work with JavaMOP 2.3 as well as JavaMOP 3.0 which is a much 
newer version of JavaMOP. We optimize \textsc{JavaMop} generated aspects by
adding statements to the particular \code{pointcut} which are responsible for
monitor creation. A \code{HashMap} named \code{contextMonitorMap} is added to
the aspect which is a mapping between a specific context (\code{long}) and an
\code{ArrayList} of monitors associated to that context. Additionaly we have
added a method named \code{getProbabilityVal()} to determine the probability of
creating a monitor in runtime depending on the number of monitors associated to
a specific context (getching from \code{contextMonitorMap}). We have also added
additional \code{pointcut} to measure evaluation parameters such as \#monitor,
\#events, time and memory consumption of the events individually.


\subsection{Context Matching}
\label{subsec:contextMatch}

Context matching codes are written inside the \code{pointcut} which are
responsible for monitor creation. Runtime method context can be determined by
looking at the \code{trace} variable (code~\ref{snippet:methodTrace})) which
represents current calling context. \code{trace} contains 

The time and memory overheads have been reported after using -converge option 
provide by the \textsc{DaCapo} benchmark suite.
\section{Evaluation}
\label{sec:evaluation}

We now describe an evaluation of our approach.

\xref{sec:evaluation:rq} lists all the research questions we address in this
paper .In~\xref{sec:evaluation:resource} and~\xref{sec:evaluation:bounded} we
describe memory consumption and time data respectively of our optimized aspects approach compared to the original \textsc{JavaMop} generated aspects. 
\xref{sec:evaluation:effectiveness} shows the amount of errors reported by our
optimized aspects.

\subsection{Research questions} 
\label{sec:evaluation:rq}

\begin{mybullet}
 \item[\textbf{RQ1}] Does our approach consume less memory?
\item[\textbf{RQ2}] Does it incur overall higher time overhead (in comparison
with the unoptimized approach)?
\item[\textbf{RQ3}] Does our approach bound worst case execution time for all
monitoring operations?
\item[\textbf{RQ4}] Does our approach effectively catch errors or
does it compromise with error reporting?
\end{mybullet}



\myparagraph{Experimental Setup} 

We ran all our experiments on a laptop with $2.5$ GHz processor and $16$ GB RAM
running OS Windows 7 64-bit. We use JVM version $8$ update $60$ with allocated heap size of $8$ GB, \aspectj\ compiler version $1.8.7$ and
\soot\ version $2.5.0$. For evaluation purpose, we use \dacapo\
benchmark version $2006-10-MR2$ \& $9.12$. We use both $\textsc{JavaMop}$ versions
$2.3$ and $3.0$ to generate aspects on top of which we wrote our optimizations.
We have considered four benchmarks from \dacapo\ benchmark suite which
are \bloat, \pmd, \chart\ and \avrora. We have ignored rest of the benchmarks as
they do not contain sufficient monitor event. For type-state property, we
consider \hasnext, \unsafeiter\ and \hashset. 



\begin{table*}[!ht]
\centering
\scriptsize
\begin{tabular}{|c|c|c|c|c||c|c|c|c||c|c|c|c|}
\hline
  \multirow{2}{*}{}                                 & 
\multicolumn{4}{c||}{HasNext}           & \multicolumn{4}{c||}{FailSafeIter}
   &    \multicolumn{4}{c|}{HashSet}
      \\ \cline{2-13}                                              
           
           
 & bloat & pmd & chart & avrora & bloat & pmd & chart & avrora& bloat
 & pmd & chart & avrora\\ \hline
 
  Original  & $20694$ & $15551$ & $7074$ & $52825$ & $23471$ & $20688$ & $6909$
  & $56873$ & $16031$ & $17017$ & - & $52895$\\\hline
 
 Optimized ($\mathcal{L}(A) = 1$K)  & $13225$ & $13891$ & $6293$ & $52857$ &
 $18406$ & $13249$ & $6960$ & $55314$ & $16681$ & $17940$ & - & $54080$\\\hline
  
  Optimized ($\mathcal{L}(A) = 30$K)  & $15161$ & $14953$ & $6794$ & $53305$ &
  - & - & - & - & - & - & - & -\\\hline
     

\end{tabular}
\caption{Runtime (ms.) of \dacapo\ benchmarks, $\mathcal{L}(A)$ denotes
\#error reported.}
\label{table:time}
\end{table*}


\begin{table*}[!ht]
\centering
\scriptsize
\begin{tabular}{|c|c|c|c|c||c|c|c|c||c|c|c|c|}
\hline
  \multirow{2}{*}{}                                 & 
\multicolumn{4}{c||}{HasNext}           & \multicolumn{4}{c||}{FailSafeIter}
   &    \multicolumn{4}{c|}{HashSet}
      \\ \cline{2-13}                                              
           
           
 & bloat & pmd & chart & avrora & bloat & pmd & chart & avrora& bloat
 & pmd & chart & avrora\\ \hline
 
  Original  & $0.85$ & $0.8$ & $0.45$ & $1.1$ & 
              $0.85$ & $0.9$ & $0.32$ & $1.2$ & 
              $0.79$ & $0.89$ & - & $1.1$\\\hline
 
 Optimized ($\mathcal{L}(A) = 1$K)  & 
 			$0.49$ & $0.49$ & $0.4$ & $0.8$ &
 			  $0.77$ & $0.52$ & $0.29$ & $0.7$ & 
 			  $0.5$ & $0.49$ & - & $0.6$\\\hline
  
  Optimized ($\mathcal{L}(A) = 30$K)  & 
  $0.52$ & $0.51$ & $0.4$ & $0.8$ &
  - & - & - & - &
   - & - & - & -\\\hline
     

\end{tabular}
\caption{Peak Memory consumption (in GB)}
\label{table:memory}
\end{table*}



\begin{table*}[!ht]
\centering
\scriptsize
\begin{tabular}{|c|c|c||c|c||c|c||c|c|}
\hline
\multicolumn{9}{|c|}{\bf\code{HasNext}}\\\hline
\multirow{3}{*}{}               & \multicolumn{2}{c||}{bloat}             & 
\multicolumn{2}{c||}{pmd}            & \multicolumn{2}{c||}{chart}      & 
\multicolumn{2}{c|}{avrora} \\\cline{2-9} 
& $\mathcal{N}(E)/\mathcal{N}(C)$  & $\mathcal{N}(A)$ &
$\mathcal{N}(E/\mathcal{N}(C))$  & $\mathcal{N}(A)$ &
$\mathcal{N}(E)/\mathcal{N}(C)$  & $\mathcal{N}(A)$ &
$\mathcal{N}(E)/\mathcal{N}(C)$  & $\mathcal{N}(A)$ 
\\ \hline
 
 Original   & $44/3$ & $1.9$M & $400/3$ & $1.94$M & $0$ & $817$ & $7.9$K$/9$&
 $898$K
 \\
 \hline Optimized ($\mathcal{L}(A) = 1$K) & $3/3$  & $10$K  & $322/3$ & $10$K 
 & $0$ & $101$ & $726/9$ & $8.2$K
 \\
 \hline Optimized ($\mathcal{L}(A) = 30$K) & $3/3$  & $110$K & $390/3$ &
 $224$K & $0$ & $817$ & $10.3$K $/9$ & $119$K
 \\
 \hline \multicolumn{9}{|c|}{\bf\code{FailSafeIter}}\\\hline
  Original  & $0$     & $1.96$M&  $0$ & $1.94$M & $0$ & $817$ & $0$& $898$K  \\
  \hline Optimized ($\mathcal{L}(A) = 1$K) & $0$ & $20$K & $0$ & $20$K & $0$ & $324$ &
 $0$ & $16.7$K \\\hline
 %Optimized ($\mathcal{L}(A) = 30$K) & $0$ & $220$K & $0$ & $449$K & $0$ &
% $324$ & $0$ & $167$K \\\hline
 %Optimized2 & & & & & & & &  \\ \hline
 \multicolumn{9}{|c|}{\bf\code{HashSet}}\\\hline
  Original  & $0$     & $11.8$K& $0$ & $7.7$ K & - & - & $0$& $8$ \\ \hline
 Optimized($\mathcal{L}(A) = 1$K) & $0$ & $5.3$K & $0$ & $408$ & - & - & $0$&
 $8$
 \\
 \hline
 %Optimized($\mathcal{L}(A) = 30$K) & $0$ & $5.3$K & $0$ & $408$ & - & - & $0$&
% $8$
 %\\
 %\hline
 %Optimized2 & & & & & & & &  \\ \hline
 
\end{tabular}
\caption{Errors reported and monitors generated for different properties.
$\mathcal{N}(E)$, $\mathcal{N}(C)$ $\mathcal{N}(A)$ and $\mathcal{L}(A)$ denote
\#error reported, \#unique contexts where errors are encountered, \#monitor
allocation and \#monitor limit respectively.}
\label{table:errorreporting1}
\end{table*}



\begin{table*}[!ht]
\centering
\scriptsize
\begin{tabular}{|c|c|c|c|c|c|c||c|c|c|c|c|c|}
\hline
\multirow{1}{*}{}                                                                
     & \multicolumn{6}{c||}{\bf Original aspect} & 
\multicolumn{6}{c|}{\bf Optimized aspects} \\ \cline{1-13} 
                                                                                 
 \multirow{3}{*}{\bf HashSet property}    & \multicolumn{2}{c|}{\bf add}  &
 \multicolumn{2}{c|}{\bf remove} & \multicolumn{2}{c||}{\bf contain} &
 \multicolumn{2}{c|}{\bf add}   &
 \multicolumn{2}{c|}{\bf remove} & \multicolumn{2}{c|}{\bf contain}         \\
 \cline{2-13} & max & min & max & min & max & min & max & min & max & min & max & min
 \\\cline{2-13}  
 &  $11$ & $1$   &   $3$     & $1$   &  $1$   & $1$   &   $1$    & $1$   &  $1$   
 & $1$ & $2$ & $1$\\\hline
 
  \multirow{3}{*}{\bf UnsafeIter property}    & \multicolumn{2}{c|}{\bf create} 
  & \multicolumn{2}{c|}{\bf update} & \multicolumn{2}{c||}{\bf next} &
 \multicolumn{2}{c|}{\bf create}   &
 \multicolumn{2}{c|}{\bf update} & \multicolumn{2}{c|}{\bf next}         \\
 \cline{2-13} & max & min & max & min & max & min & max & min & max & min & max & min
 \\\cline{2-13}  
 &  $1592$ & $1$   &   $65$     & $1$   &  $5$   & $1$   &   $37$    & $1$   & 
 $2$ & $1$ & $80$ & $1$\\\hline
 
   \multirow{3}{*}{\bf HasNext property}    & \multicolumn{3}{c|}{\bf hasNext} 
   & \multicolumn{3}{c||}{\bf next} &
 \multicolumn{3}{c|}{\bf hasNext}   & \multicolumn{3}{c|}{\bf next} \\
 \cline{2-13}
 
 & \multicolumn{1}{c|}{max} & \multicolumn{2}{c|}{min} &  
 \multicolumn{1}{c|}{max} & \multicolumn{2}{c||}{min} &
 \multicolumn{1}{c|}{max} & \multicolumn{2}{c|}{min} &
 \multicolumn{1}{c|}{max} & \multicolumn{2}{c|}{min}  \\\cline{2-13}  
 &   \multicolumn{1}{c|}{$944$} & \multicolumn{2}{c|}{$1$}   &  
 \multicolumn{1}{c|}{$89$} & \multicolumn{2}{c||}{$1$} &  
 \multicolumn{1}{c|}{$3$}   & \multicolumn{2}{c|}{$1$}   & 
 \multicolumn{1}{c|}{$10$}   & \multicolumn{2}{c|}{$1$}   \\\hline
 
 

\end{tabular}
\caption{Comparison of event times(ms) of \dacapo\ \bloat.}
\label{table:eventTime}
\end{table*}

%%%%%%%%%%%%%%%%%%%%%%%%%%%%%%%%%%%%%%%%%%%%%%%%%%%%%%%%%%%%%%%%


\subsection{Resource Consumption}
\label{sec:evaluation:resource}

% \note{Add stuff about \\
% Q1 Does our approach consume less memory?
% Q4 Does it incur overall higher time overhead (in comparison with the 
% unoptimized approach)?\\
% Web server experiment?
% }

Here we address \textbf{RQ1} and \textbf{RQ2} (refere~\xref{sec:evaluation:rq})
which talk about resource consumption of our approach compared to the aspects
generated by \textsc{JavaMop}. Resource consumption data is represented in terms
of runtime and main memory consumption, which is tabulated in the
Tables~\ref{table:time} and ~\ref{table:memory}. Table~\ref{table:time}
tabulates all the run time of benchmark execution in ms. For comparison, we use
the aspect generated by \javamop\ which is denoted as \emph{Original} in the
table. \emph{Optimized($\mathcal{L}(a) = 1K$)} and
\emph{Optimized($\mathcal{L}(a) = 30K$)} denotes to the optimized monitors with
$1$K and $30$K limit on the size of monitor pool.
It is evident from Table~\ref{table:time} that most of the cases the benchmark
executed with the optimized monitor with less of almost equal time. That
concludes that our optimizations do not incurs any extra overhead.
Table~\ref{table:memory} represents the memory consumption data in similar way.
The memory consumption of optimized aspects are always less than the original
aspects generated by \javamop\ as the former one spawns significant lesser number
of monitors(see data in Table~\ref{table:errorreporting1}).


\subsection{Bounded Execution Time}
\label{sec:evaluation:bounded}

% \note{Add stuff about\\
% Q2. Does our approach bound worst case execution time for all monitoring 
% operations?
% }
Here we address \textbf{RQ3} \ie bound on worst case execution time. We executed
\bloat benchmarks with both \javamop generated aspects and our optimized aspect
and tabulated the execution time corresponding to each events in
Table~\ref{table:eventTime}. In all of the cases, we notice significant
reduction of event time.In \hashset\ property, we experience $99\%$ reduction of
maximum event time for \texttt{add} event.



\subsection{Effectiveness of the approach}
\label{sec:evaluation:effectiveness}

% \note{Add stuff about\\
% Q3. Does our approach effectively catch errors or does it compromise with error 
% reporting?
% }
Table~\ref{table:errorreporting1} addresses \textbf{RQ4}, \ie effectiveness of
our optimized monitoring approach. The above mentioned table provides a
comparative data of number of reported errors($\mathcal{N}(E)$) and number of
monitors allocated ($\mathcal{N}(M)$) in three scenarios : \textbf{(a)}
\emph{Original}: aspect generated by \javamop,
\textbf{(b)}\emph{Optimized($\mathcal{L}(a) = 1K$)} optimized aspect with $1$K
monitor bound on global monitor pool and \textbf{(c)}
\emph{Optimized($\mathcal{L}(a) = 30K$)} optimized aspect with $1$K monitor
bound on global monitor pool. Across all the benchmarks and properties, we
experience significant reduction of allocated monitors which resulted in lesser
memory consumption (refer~\xref{sec:evaluation:resource}). In $\mathcal{N}(E)$
column, it can be seen that our optimized monitors report lesser number of
errors. But all of these errors are not unique. Closer look to the method
context trace revealed that in case of \bloat and \avrora, all the errors
reported by the \javamop generated aspects are generated from $3$ and $9$ unique
contexts respectively. Our optimized monitor did not missed any unique error
report, rather exclude some of the ambiguous ones.

\subsection{Discussion}
\label{sec:discussion}

% \note{Add stuff about \\
% Threats to validity,\\Strengths \& Weaknesses
% }


\section{Discussion}
\label{sec:discussion}

\subsection{Threats to validity}
\label{sec:discussion:ttv}

Our study is restricted to four \dacapo benchmarks and three \java standard
library properties. Even though these combinations have been found to be
challenging to monitor in the past, it is possible that the combinations are not
representative and results of the study will change if we include more
combinations. We intend to this extended study in the future.

We used \javamop\ $2.1.2$ as a baseline tool for our prototype implementation. 
However, the results of the study may change if we use a different tool or a 
more recent version of JavaMOP. However, using an older version of \javamop had 
an advantage of being simpler and easier to understand which allowed us to 
ensure that our optimizations do not interfere with \javamop's optimizations. 
Moreover, our goal is not compare the performance with JavaMOP, but only to show 
that our technique is complementary to JavaMOP optimizations and can be used to 
extend \javamop which would add further to its effectiveness.

The choices of hardware and software platforms, in particular, the server 
settings may influence the results. In the future we plan to repeat the study on 
a variety of platforms to understand their impact on the results.

\subsection{Strengths \& Weaknesses}
\label{sec:discussion:snw}

The results presented in Section~\ref{subsec:results} and the discussion in 
Section~\ref{subsubsec:efficiencyanddeterminism} indicate that the approach can 
i) effectively control the memory needs, ii) does not compromise much with the 
error reporting, iii) provides worst-case bounds to all monitoring operations, 
and iv) does not produce false positives which preserves its completeness. 
However, current implementation does not efficiently control the runtime 
overhead. Table~\ref{table:runtimeoverhead} provides the runtime overheads for 
the same DaCapo benchmarks and program properties used previously in this 
section.
% \note{Add stuff about \\
% Threats to validity,\\Strengths \& Weaknesses
% }


\section{Related Work}

There is a substantial work performed on runtime monitoring to ensure that the 
running software is in consistent state. These checks are performed either by 
reading the events generated by the software and then applying some logic that 
models the predefined rules or properties. Alternatively, the checks are 
performed implicitly by the programs when some extra code is inserted inside the 
programs. In majority of these cases the system is assumed to have abundant 
resources in terms of memory and have no time constraints. These assumptions do 
not hold for resource-constrained systems such as real-time embedded systems 
which pose serious challenges to runtime monitoring. We briefly discuss here 
some of the notable related work in runtime monitoring in general and runtime 
monitoring for embedded systems in particular.

\textbf{\textit {Approaches for Real-Time Systems}} A lot of work in the past 
has focussed on reducing time overhead of runtime monitoring. However, real-time 
embedded systems demand time-predictable or deterministic runtime monitoring. 
The challenge is in scheduling the monitoring activities so that they do not 
interfere with the software operation and do not violate the software's 
nonfunctional properties. Some approaches depend on event sampling and optimized 
time-triggering \cite{ArafaKF13, NavabpourBF12, WuKBF13}. Other approaches 
include predictable monitoring that provide bounds on detection latency 
\cite{ZhuDG09, ZhuGD10}. These approaches are effective, however, they do not 
target general finite state properties that might be related to multiple 
objects. Moreover, many of these approaches are unsuitable for inline 
monitoring. Our approach, in contrast, tries to support inline monitoring of 
finite state monitoring.

\textbf{\textit {Sampling-based approach}} \ignore {In [7], the authors have 
presented a specialized runtime environment, Quality Virtual Machine (QVM), 
targeted towards defect detection and diagnosis in production systems. It tracks 
safety properties, Java assertions and heap properties for violations. A novel 
overhead manager that enforces a user specified overhead budget for quality 
checks is a unique feature of QVM.To reduce the analysis overhead, QVM performs 
property-guided sampling ensuring the sampling is not done randomly. Another 
feature supported by QVM is object-centric sampling which allows to sample at 
the object instance level. If an object is sampled at allocation time, a bit is 
set in object header to mark the object as tracked. QVM tracks the number of 
times a typestate property has been violated and if the number passes a 
specified threshold, it starts recording typestate history. Typestate history of 
an object, the abstraction of method call invocations performed during execution 
with the 
object as reciever, gives the information about the way the object was used in 
the program that violates the property. QVM is implemented on top of JVM, which 
is an advantage but at the same time it comes with cost of non-portability. QVM 
can effectively monitor large applications but supports single-object typestate 
properties. This work is by far close to our approach but the QVMs approach is 
unsound as it cannot detect violations for objects that are dropped. However, we 
have ensured the soundness or completeness of our approach even after sampling 
the objects to reduce the overhead.}
Researchers have presented approaches that are based on sampling object space 
\cite{Arnold:OOPSLA08}, sampling time \cite{BartocciGKSSZS12, StollerBSGHSZ11},
and sampling properties \cite{dwyer08ase}. 
Among these approaches, the one presented by Arnold et al.  
\cite{Arnold:OOPSLA08} is closest to us in its spirit. They develop Quality 
Virtual Machine (QVM) that tracks safety properties, Java assertions and heap 
properties for violations. It also has an overhead manager to enforce a user 
specified overhead budget. Even though effective for general-purpose 
applications, QVM is not designed for real-time system requirements. It is not 
easily portable and it apparently tracks only single-object properties. Similar 
to our approach none of these systems sacrifices completeness for sampling. 
However, unlike our approach none of these systems reuse monitors.
\\
\textbf{\textit {Aspect-based and similar monitoring approaches}} A number of 
finite state runtime monitoring tools including Tracematches \cite{Allan:OOPSLA05},
JavaMOP \cite{meredith-2008}, and MARQ \cite{Reger2015}
have been developed to detect violations of typestate properties. 
JavaMOP in particular has been expressive in terms of its specification power 
and supports various formalisms ranging from regular expressions to context-free 
grammars. It converts its specification to an aspect which is then woven into a 
program to be monitored. Due to the flexibility in its architecture  JavaMOP 
allows various optimizations to be performed making it an efficient monitoring 
tool. In spite of this efficiency, for certain program and property 
combinations, all of these tools incur heavy and unbounded overheads in terms of 
memory and time. These scenarios act as the motivation to our research. Various 
approaches \cite{Avgustinov:2007, luo-2014, meredith-2008, Purandare:2013} have been proposed to control the memory as well as avoid 
unnecessary monitoring that cannot lead to any errors. However, in spite of 
these 
efforts and the effectiveness of their approaches, runtime monitoring still 
remaining challenging particularly finite state properties.


\ignore{
\\In [15], Avgustinov et al. identify two optimizations, a combination of leak 
elimination and indexing, that analyse the tracematch's specified declarative 
pattern and not the monitored code itself. One of the optimizations is 
discarding the unnecessary monitors which could improve the memory behaviour and 
avoids space leaks. The authors have identified that the performance of garbage 
collector affects the optimisation. In other words, if a program runs with 
smaller heap, the garbage collector runs all the time and thus perform better 
than programs with larger heap. The second optimisation is a form of indexing of 
the set of monitors for their efficient tracking. The indexing also eliminates 
the crucial dependency on garbage collector performance. On evaluating with a 
few benchmark-property combinations, it shows significant speed ups in many 
cases but similar or even worse performance in few other combinations.
\\
Alternatively, in  the authors propose garbage collection of unnecessary monitor 
instances to prune the unnecessary monitors based on static analysis of the 
monitored property. If an object is garbage collected, then the monitor instance 
binded to that object becomes unnecessary. The information of garbage collection 
of parameter objects is propogated and the unnecessary monitors are removed 
lazily to avoid creating undue overhead. The RV system builds upon the Indexing 
Tree Technique of the JavaMOP system to access the objects and related monitors. 
Garbage collection of unnecessary monitors reduces memory usage and the time 
required to update monitor instances. The average overhead of RV on evaluating 
with Dacapo Benchmarks [11] for various safety properties is lower than JavaMOP 
and Tracematches. We, on the contrary, are not using static analysis but 
maintaining a pool of monitor instances that will be reused to ensure less 
memory usage and at the same time check for typestate property violations.
}

\textbf{\textit {Hybrid approaches}} Several hybrid approaches have been 
proposed by research that combine static program analysis with dynamic program 
analysis \cite{Bodden:ECOOP07, bodden:fse08, bodden:icse10, Bodden:RV10, Dwyer:ASE07, Purandare:OOPSLA10}.
The static component of the analysis essentially filters the 
program points that do not need to be monitored. These approaches are effective, 
and have been found to control time overheads up to certain extent but also have 
been found to be unsatisfactory in many cases. Moreover, they have not been 
effective in controlling the space requirements. Moreover, none of these 
approaches can provide bounds on the execution times of monitoring operations. 
The usage of static analysis can often be unsatisfactory due to numerous false 
positives and can be time-consuming adversely impacting the development and 
analysis of the software.
\\


\section{Conclusion and Future Work}
\label{sec:conclusion}

We presented a novel technique that explores the trade-off between efficiency 
and determinism, and reported violations. It samples the objects for monitoring 
in order to reduce the memory overhead. At the same time it strives to catch all 
distinct property violations that an optimized program would catch. Our 
approach provides worst-case execution time bounds for all monitoring 
operations.   As indicated by the study, our approach can reduce the monitor 
instances without compromising much with the error reporting.

Previous approaches that have been proposed by researchers to employ runtime 
monitoring in real-time embedded systems have been either outline or offline, 
and many of them target single state properties. General monitoring approaches 
do not take into consideration the limited availability of resources. These 
approaches are not suitable for resource-constrained systems. We hope that the 
novel features of our approach would help employ finite state inline monitoring 
in the domain of real-time embedded systems. The ability to perform inline 
monitoring also offers opportunities to perform error recovery and better error 
diagnosis on the fly.

In the future, we intend to develop a fully automated implementation and extend 
our study to include a few more challenging benchmarks and property 
combinations. We would like to explore static analysis based approaches to 
access program execution context and reduce the runtime overhead.


\clearpage

\raggedright
\small

\bibliographystyle{acm}
\bibliography{paper}

% \theendnotes

%%\justifying
%%\appendix
%%\input{sections/appendix}

\end{document}
