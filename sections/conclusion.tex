\section{Conclusion and Future Work}
\label{sec:conclusion}

We presented a novel technique that explores the trade-off between efficiency 
and determinism, and reported violations. It samples the objects for monitoring 
in order to reduce the memory overhead. At the same time it strives to catch all 
distinct property violations that an optimized program would catch. Our 
approach provides worst-case execution time bounds for all monitoring 
operations.   As indicated by the study, our approach can reduce the monitor 
instances without compromising much with the error reporting.

Previous approaches that have been proposed by researchers to employ runtime 
monitoring in real-time embedded systems have been either outline or offline, 
and many of them target single state properties. General monitoring approaches 
do not take into consideration the limited availability of resources. These 
approaches are not suitable for resource-constrained systems. We hope that the 
novel features of our approach would help employ finite state inline monitoring 
in the domain of real-time embedded systems. The ability to perform inline 
monitoring also offers opportunities to perform error recovery and better error 
diagnosis on the fly.

In the future, we intend to develop a fully automated implementation and extend 
our study to include a few more challenging benchmarks and property 
combinations. We would like to explore static analysis based approaches to 
access program execution context and reduce the runtime overhead.
