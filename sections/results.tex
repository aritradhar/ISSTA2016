\section{Evaluation}
\label{sec:evaluation}

We list below the key research questions about the effectiveness of our approach 
that we address in this evaluation.
\begin{mybullet}
\item[\textbf{RQ1}] Does it consume less memory?
\item[\textbf{RQ2}] Does it incur higher execution time overhead (as compared to 
the unoptimized approach)?
\item[\textbf{RQ3}] Does it bound the worst case execution time for all 
monitoring operations?
\item[\textbf{RQ4}] Does it effectively catch errors or does it compromise with 
error reporting?
\end{mybullet}

\noindent Specifically, in~\xref{sec:evaluation:resource} 
and~\xref{sec:evaluation:bounded} we evaluate our approach both memory and worst 
case execution overheads respectively. In~\xref{sec:evaluation:effectiveness} we 
demonstrate the efficacy of our approach in removing redundant errors.

\myparagraph{Experimental Setup} All our experiments were performed on a laptop 
provisioned with $2.5$ GHz processor, $16$ GB RAM and running $64$-bit Windows 
$7$. We use JVM v$8$ with allocated heap size of $8$ GB, \aspectj\ compiler 
v$1.8.7$ and \soot\ v$2.5.0$. For evaluation, we use \dacapo\ benchmark version 
$2006-10-MR2$ \& $9.12$. We experiment with both $\textsc{JavaMop}$ v$2.3$ and 
v$3.0$ to generate aspects, above which we wrote our optimizations. We 
considered four benchmarks from \dacapo\ benchmark suite, namely \bloat, \pmd, 
\chart\ and \avrora. We ignored other benchmarks in the suite as they do not 
contain sufficient monitor events. We consider \hasnext, \unsafeiter\ and 
\hashset\ as the type-state property of interest to be monitored.

\begin{table*}[!ht]
\centering
\scriptsize
\begin{tabular}{|c|c|c|c|c||c|c|c|c||c|c|c|c|}
\hline
  \multirow{2}{*}{}                                 & 
\multicolumn{4}{c||}{HasNext}           & \multicolumn{4}{c||}{FailSafeIter}
   &    \multicolumn{4}{c|}{HashSet}
      \\ \cline{2-13}                                              
           
           
 & bloat & pmd & chart & avrora & bloat & pmd & chart & avrora& bloat
 & pmd & chart & avrora\\ \hline
 
  Original  & $20694$ & $15551$ & $7074$ & $52825$ & $23471$ & $20688$ & $6909$
  & $56873$ & $16031$ & $17017$ & - & $52895$\\\hline
 
 Optimized ($\mathcal{L}(A) = 1$K)  & $13225$ & $13891$ & $6293$ & $52857$ &
 $18406$ & $13249$ & $6960$ & $55314$ & $16681$ & $17940$ & - & $54080$\\\hline
  
  Optimized ($\mathcal{L}(A) = 30$K)  & $15161$ & $14953$ & $6794$ & $53305$ &
  - & - & - & - & - & - & - & -\\\hline
     

\end{tabular}
\caption{Runtime (ms.) of \dacapo\ benchmarks, $\mathcal{L}(A)$ denotes
\#error reported.}
\label{table:time}
\end{table*}


\begin{table*}[!ht]
\centering
\scriptsize
\begin{tabular}{|c|c|c|c|c||c|c|c|c||c|c|c|c|}
\hline
  \multirow{2}{*}{}                                 & 
\multicolumn{4}{c||}{HasNext}           & \multicolumn{4}{c||}{FailSafeIter}
   &    \multicolumn{4}{c|}{HashSet}
      \\ \cline{2-13}                                              
           
           
 & bloat & pmd & chart & avrora & bloat & pmd & chart & avrora& bloat
 & pmd & chart & avrora\\ \hline
 
  Original  & $0.85$ & $0.8$ & $0.45$ & $1.1$ & 
              $0.85$ & $0.9$ & $0.32$ & $1.2$ & 
              $0.79$ & $0.89$ & - & $1.1$\\\hline
 
 Optimized ($\mathcal{L}(A) = 1$K)  & 
 			$0.49$ & $0.49$ & $0.4$ & $0.8$ &
 			  $0.77$ & $0.52$ & $0.29$ & $0.7$ & 
 			  $0.5$ & $0.49$ & - & $0.6$\\\hline
  
  Optimized ($\mathcal{L}(A) = 30$K)  & 
  $0.52$ & $0.51$ & $0.4$ & $0.8$ &
  - & - & - & - &
   - & - & - & -\\\hline
     

\end{tabular}
\caption{Peak Memory consumption (in GB)}
\label{table:memory}
\end{table*}



\begin{table*}[!ht]
\centering
\scriptsize
\begin{tabular}{|c|c|c||c|c||c|c||c|c||c|c|}
\hline
\multicolumn{11}{|c|}{\bf\code{HasNext}}\\\hline
\multirow{3}{*}{}               & \multicolumn{2}{c||}{bloat}             & 
\multicolumn{2}{c||}{pmd}            & \multicolumn{2}{c||}{chart}      & 
\multicolumn{2}{c||}{avrora} & \multicolumn{2}{c|}{Synthetic}\\\cline{2-11} 
& $\mathcal{N}(E)/\mathcal{N}(C)$  & $\mathcal{N}(A)$ &
$\mathcal{N}(E/\mathcal{N}(C))$  & $\mathcal{N}(A)$ &
$\mathcal{N}(E)/\mathcal{N}(C)$  & $\mathcal{N}(A)$ &
$\mathcal{N}(E)/\mathcal{N}(C)$  & $\mathcal{N}(A)$ &
$\mathcal{N}(E)/\mathcal{N}(C)$  & $\mathcal{N}(A)$ 
\\ \hline
 
 Original   & $44/3$ & $1.9$M & $400/3$ & $1.94$M & $0$ & $817$ & $7.9$K$/9$&
 $898$K & $6$M$/3$ & $3$M
 \\
 \hline Optimized ($\mathcal{L}(A) = 1$K) & $3/3$  & $10$K  & $322/3$ & $10$K 
 & $0$ & $101$ & $726/9$ & $8.2$K & $7.4$K$/3$ & $7.4$K
 \\
 \hline Optimized ($\mathcal{L}(A) = 30$K) & $3/3$  & $110$K & $390/3$ &
 $224$K & $0$ & $817$ & $10.3$K $/9$ & $119$K & $100$K$/3$ & $100$K
 \\\hline 
 \multicolumn{11}{|c|}{\bf\code{FailSafeIter}}\\\hline
  Original & $0$ & $1.96$M&  $0$ & $1.94$M & $0$ & $817$ & $0$& $898$K &
  -&-\\\hline Optimized ($\mathcal{L}(A) = 1$K) & $0$ & $20$K & $0$ & $20$K &
  $0$ & $324$ & $0$ & $16.7$K &- & -\\\hline
 %Optimized ($\mathcal{L}(A) = 30$K) & $0$ & $220$K & $0$ & $449$K & $0$ &
% $324$ & $0$ & $167$K \\\hline
 %Optimized2 & & & & & & & &  \\ \hline
 \multicolumn{11}{|c|}{\bf\code{HashSet}}\\\hline
  Original  & $0$     & $66.8$K& $0$ & $6.8$M & - & - & $0$& $106$  & -&-\\
  \hline Optimized($\mathcal{L}(A) = 1$K) & $0$ & $2.7$K & $0$ & $10$K & - & - & $0$&
 $99$  & -&-\\ \hline

\end{tabular}
\caption{Errors reported and monitors generated for different properties.
$\mathcal{N}(E)$, $\mathcal{N}(C)$ $\mathcal{N}(A)$ and $\mathcal{L}(A)$ denote
\#error reported, \#unique contexts where errors are encountered, \#monitor
allocation and \#monitor limit respectively.}
\label{table:errorreporting1}
\end{table*}



\begin{table*}[!ht]
\centering
\scriptsize
\begin{tabular}{|c|c|c|c|c||c|c|c|}
\hline
\multirow{1}{*}{} & \multirow{1}{*}{}                                                                
     & \multicolumn{3}{c||}{\bf Original aspect} & 
\multicolumn{3}{c|}{\bf Optimized aspects} \\ \cline{1-8} 
                                                                                 
 \multirow{5}{*}{\bf \texttt{HashSet} property}  &   & \bf add  &
 \bf remove & \bf contain &
 \bf add  &
 \bf remove& \bf contain       \\  \cline{2-8} 
 
 & \bloat & 13 & 2 & 2 & 2 & 2 & 2 \\\cline{2-8} 
 & \pmd   & 84 & - & 18& 3 & - & 22 \\\cline{2-8} 
 & \chart & - & - & - & - & - & - \\\cline{2-8} 
 & \avrora & 2 & - & 5 & 1 & - & 1 \\\cline{1-8} \hline
 
                                                                                 
 \multirow{5}{*}{\bf \texttt{FailSafeIter} property}  &   & \bf create  &
 \bf update & \bf next &
 \bf create  &
 \bf update & \bf next       \\  \cline{2-8} 
 
 & \bloat & 322 & 1 & 2 & 6 & 1 & 2 \\\cline{2-8} 
 & \pmd   & 147 & 1 & 12& 4 & 1 & 12 \\\cline{2-8} 
 & \chart & 1 & 5 & 1 & 1 & 5 & 1 \\\cline{2-8} 
 & \avrora & 275 & 29 & 1 & 12 & 8 & 1 \\\cline{1-8} \hline


                                                                                 
 \multirow{5}{*}{\bf \texttt{HasNext} property}   & &
 \multicolumn{2}{c|}{\bf hasnext} & \bf next &
 \multicolumn{2}{c|}{\bf hasnext} & \bf next       \\  \cline{2-8} 
 
 & \bloat  &  \multicolumn{2}{c|}{81}  & 9  & \multicolumn{2}{c|}{1} & 2
 \\\cline{2-8} 
 & \pmd    &  \multicolumn{2}{c|}{222}  & 2 & \multicolumn{2}{c|}{1}  & 1
 \\\cline{2-8} 
 & \chart  &  \multicolumn{2}{c|}{2}  & 5  & \multicolumn{2}{c|}{1}  & 5
 \\\cline{2-8} 
 & \avrora &  \multicolumn{2}{c|}{75} & 1  & \multicolumn{2}{c|}{1}  & 1
 \\\cline{1-8} \hline
 

\end{tabular}
\caption{Comparison of event times(ms) of \dacapo\ benchmarks.}
\label{table:eventTime}
\end{table*}

%%%%%%%%%%%%%%%%%%%%%%%%%%%%%%%%%%%%%%%%%%%%%%%%%%%%%%%%%%%%%%%%


\subsection{Resource Consumption}
\label{sec:evaluation:resource}

% \note{Add stuff about \\
% Q1 Does our approach consume less memory?
% Q4 Does it incur overall higher time overhead (in comparison with the 
% unoptimized approach)?\\
% Web server experiment?
% }

In this section, we answers questions \textbf{RQ1} and \textbf{RQ2} to 
understand the resource consumption of our approach. We execute the four 
\dacapo\ benchmarks and measure the total time for execution and memory 
consumption. Tables~\ref{table:time} and~\ref{table:memory} list the results.
% Table~\ref{table:time} tabulates all the run time of benchmark execution in ms. 
We compare the performance against the aspects generated by \javamop, which is 
denoted as \emph{Original} in the table. \emph{Optimized($\mathcal{L}(a) = 1K$)} 
and \emph{Optimized($\mathcal{L}(a) = 30K$)} denote the optimized monitors with 
$1$K and $30$K limit on the size of monitor pool. We observe that for most cases 
the benchmarks with the optimized monitor execute nearly as fast as the 
unoptimized ones, which denotes that our optimizations do not incur any 
significant overhead. Table~\ref{table:memory} represents the memory 
consumption for the benchmarks. The memory consumption of the optimized aspects 
is always less than the original aspects generated by \javamop. This low 
overhead is because our approach spawns significantly less number of 
monitors (see Table~\ref{table:errorreporting1}).

\subsection{Bounded Execution Time}
\label{sec:evaluation:bounded}

% \note{Add stuff about\\
% Q2. Does our approach bound worst case execution time for all monitoring 
% operations?
% }
We now address \textbf{RQ3}, \ie\ bound on worst case execution time. We
experimet on the \dacapo\ benchmarks, and observe the execution time with both
\javamop\ generated aspects and our optimized aspect. 
Table~\ref{table:eventTime} lists  the execution time corresponding to each
events. We observe significant reduction of event time across all cases.
Specifically, for the \unsafeiter\ property, we observe $\sim$$99\%$ reduction
of maximum event time for the \texttt{create} event in \bloat.

\subsection{Effectiveness of the approach}
\label{sec:evaluation:effectiveness}

% \note{Add stuff about\\
% Q3. Does our approach effectively catch errors or does it compromise with error 
% reporting?
% }

We test the effectiveness of our approach (\ie\ \textbf{RQ4}) by determining 
if our approach missed out on errors due to the less monitors generated. We 
measure the number of reported errors($\mathcal{N}(E)$) and the number of 
monitors allocated ($\mathcal{N}(M)$) under three scenarios : \textbf{(a)}
\emph{Original}: aspects generated by \javamop, 
\textbf{(b)}\emph{Optimized($\mathcal{L}(a) = 1K$)} optimized aspect with $1$K 
monitor bound on global monitor pool, and \textbf{(c)} 
\emph{Optimized($\mathcal{L}(a) = 30K$)} optimized aspect with $1$K monitor
bound on global monitor pool. Table~\ref{table:errorreporting1} lists the 
results of our optimized monitoring approach. We observe that across all the 
benchmarks and properties, our approach enables significant reduction of 
allocated monitors, which result in lesser memory consumption 
(refer~\xref{sec:evaluation:resource}). Note that even though our optimized 
monitors report lesser number of errors. A closer inspection of the context 
trace reveals that in case of \bloat\ and \avrora, all the errors
reported by the \javamop\ generated aspects are generated from just $3$ and $9$ 
unique contexts respectively. Thus, our optimized monitors did not miss out on 
any unique error report; they rather exclude duplicate errors. Additionally we
have also developed an synthetic test case which emulates a server processing a
large number of queries. The test case consists of three methods and all of them
violate \hasnext\ property. Example of one such method of the synthetic test
case is described in the Code~\ref{synthetic}.

\lstset{escapeinside={/*@}{@*/}, language=Java , caption=\bf
Synthetic testcase., label=synthetic}
\begin{figure}[h]
\begin{lstlisting}
public foo(List a){
 List b ;
 for(int i = 0; i < 1000000; i++){
  Iterator it = a.iterator();
  b.add(it.next() * it.next());//hasnext violation
 }
 bar(b);
}
public bar(List b){...}
\end{lstlisting}
\end{figure}

We intentionally keep number of monitors very high
to simulate large amount of query processing and effectiveness of our optimized
approach. Similar to \dacapo\ benchmarks, the synthetic test case also produces
consistent result. The aspect generated by \javamop\ produces $3$M monitors and
total of $6$M error reports. All of the $6$ millions errors are generated from
$3$ unique context. Our optimized aspect generated $7.4$K and $100$K monitors
when the limit on global monitor pool is assigned to $1$K and $30$K respectively.
Similar phenomenon can be observed in number of error reported by the optimized
aspect. The optimized version able to prune many ambiguous error reports and
reduces to $7.4$K and $100$K when the limit on global monitor pool is assigned to
$1$K and $30$K respectively

