\section{Discussion}
\label{sec:discussion}

\subsection{Threats to validity}
\label{sec:discussion:ttv}

Our study is restricted to four \dacapo benchmarks and three \java standard
library properties. Even though these combinations have been found to be
challenging to monitor in the past, it is possible that the combinations are not
representative and results of the study will change if we include more
combinations. We intend to this extended study in the future.

We used \javamop\ $2.1.2$ as a baseline tool for our prototype implementation. 
However, the results of the study may change if we use a different tool or a 
more recent version of JavaMOP. However, using an older version of \javamop had 
an advantage of being simpler and easier to understand which allowed us to 
ensure that our optimizations do not interfere with \javamop's optimizations. 
Moreover, our goal is not compare the performance with JavaMOP, but only to show 
that our technique is complementary to JavaMOP optimizations and can be used to 
extend \javamop which would add further to its effectiveness.

The choices of hardware and software platforms, in particular, the server 
settings may influence the results. In the future we plan to repeat the study on 
a variety of platforms to understand their impact on the results.

\subsection{Strengths \& Weaknesses}
\label{sec:discussion:snw}

The results presented in Section~\ref{subsec:results} and the discussion in 
Section~\ref{subsubsec:efficiencyanddeterminism} indicate that the approach can 
i) effectively control the memory needs, ii) does not compromise much with the 
error reporting, iii) provides worst-case bounds to all monitoring operations, 
and iv) does not produce false positives which preserves its completeness. 
However, current implementation does not efficiently control the runtime 
overhead. Table~\ref{table:runtimeoverhead} provides the runtime overheads for 
the same DaCapo benchmarks and program properties used previously in this 
section.
% \note{Add stuff about \\
% Threats to validity,\\Strengths \& Weaknesses
% }

