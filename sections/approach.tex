\section{Memory-Efficient and Deterministic Monitoring}
\label{sec:approach}

Our monitoring approach strives to reduce the memory requirements of finite state monitoring. At the same time it also tries to make monitoring more deterministic. The approach is based on the following mechanisms:

\paragraph{Context-based sampling} We use context-based sampling to control the number of monitors. The motivation for this approach comes from our observation that the monitors created at the same creation site and under similar context tend to go through a similar life cycle. In other words, these monitors are more likely to show a redundant behavior. Hence, the heuristic applied for this sampling is based on the program execution context. It works by i) identifying the monitor creation sites which are specified as creation events in the monitoring specification, and then by ii) obtaining the call-stack to understand the execution context, and finally by iii) making a decision about the allocation of the monitor based on the number of times this context was seen in the past. More often the monitoring system has seen the context, less likely it is to allocate a monitor. The programming context that we consider is limited to the the top three methods in the call stack along with the program counter.

\paragraph{Fixed-size global pool of monitors} We limit the total number of monitors that our monitoring approach would generate by creating a small monitor-pool of fixed size prior to the program execution and then maintaining it during the execution. In short our approach reuse of monitors. The monitors after their usage can be returned back to the system. Moreover, if the pool runs out of monitors and the system needs a new monitor, a monitor currently being used is forcefully reclaimed and made available for the reallocation. The heuristic that we use to reclaim a monitor is based on the observation that the program events are often temporally separated. In other words, in an execution trace, the events related to a same receiver object are likely to be found together. We exploit this observation by making the last used potentially active monitor available for reallocation. This also means that if the object that is removed from tracking observes another event it will not be tracked, and in case the event had otherwise generated an error, the approach would now miss it. Potentially missing a few errors is a cost of our optimized monitoring approach, however, by using smart heuristics we can minimize \textit{false negatives}.

\paragraph{Fixed-size local pool of monitors for individual objects} For properties related to multiple objects such as \texttt{UnsafeIterator} property, an object may can get associated with several monitors in its life-time. In this case, an event that performs an operation on the object results in the state of every associated monitor getting updated. Hence, handling such events may become \textit{nondeterministic} as far as timing requirements are concerned. We consider a monitoring behaviour to be deterministic if we can calculate the worst-case time taken to handle every monitoring event. Binding the number of monitors associated with an object allows us to limit the worst case execution time for any monitoring operation associated with that object. Our approach implements this constraint by allocating a fixed-size local pool of monitors to individual objects.

Section~\ref{subsec:outline} gives the outline of our approach, and Section~\label{subsec:algorithm} presents the algorithm.

\subsection{Monitoring Approach}
\label{subsec:outline}

\begin{figure}[t]
\centering
  \includegraphics[scale=0.4, trim= 2cm 1cm 0 1cm]{./images/schematic.pdf}
  \caption[Schematic of Memory-efficient and Deterministic Monitoring System]{Schematic of Memory-efficient and Deterministic Monitoring System.}
  \label{fig:schematic}
\end{figure}

Figure~\ref{fig:schematic} shows the main elements of our approach. The program under execution generates events parameterized by objects which are accepted by our monitoring system, and each one of them is handed over to the monitor allocation component. This component based on the event type and some heuristics makes a decision about whether to allocate a monitor or not. For convenience, we specify the program points that generate creation events explicitly, so that the approach can leverage this information to make the decision. If the event is of creation type, the monitoring system checks the execution context and its tracking history to see is the context was already sen in the past, and if yes, then with that frequency. The system probabilistically skips the monitor allocation phase for frequently seen contexts. The contexts are maintained as trees in which a new context adding a new branch or branches and leaf node. Every leaf node uniquely defines a context and it keeps information about the number of times the context was seen. The system can be confi

In case the system chooses to allocate a monitor, it consults the global and the local monitor pools to see if a fresh monitor can be allocated or the the local  If allocation is required it consults the global pool as well as the local pool to see if a fresh monitor is available or an old one needs to be reused. This mechanism currently is implemented as a circular array which naturally preserves the chronological ordering. In case an old one is chosen for monitoring, that monitor is first removed from the existing local pools of the corresponding objects, and then reallocated to the new object under consideration.

The tracking of monitors for the non-creation events is similar to the conventional model. The only difference is that if no monitor exists in the maps corresponding to objects associated with an event, then monitoring is completely skipped for that event. It is assumed that the monitors corresponding to the associated objects were never created by the system. Skipping monitoring is a direct saving in terms of execution time.




\begin{algorithm}[t]
                      % enter the algorithm environment
\caption[Algorithm]{Monitoring Algorithm. $\phi$ = ($Q$,$\Sigma$,$\delta$,$q_{0}$,$F$), $e=(l,b)$ where e is an event and $l \in L$ be a set of associated objects and $b \in \Sigma$}          % give the algorithm a caption
\label{alg1}                           % and a label for \ref{} commands later in the document
\begin{algorithmic}[1]                  
   %\STATE \textbf{let} \textit{O} be the set objects that receive events
   \STATE \textbf{let} $\Sigma_{c} \in \Sigma$ be the set of creation symbols
   \STATE \textbf{let} \textit{threshold} : $\mathcal{N} \to \mathcal{R}$ be a function generating a threshold value
   \STATE \textbf{let} \textit{A} be the global circular array of monitors
   \STATE \textbf{let} \textit{ObjsMons} : $O \to \textit{MS}$ be a function
   \STATE \textbf{let} \textit{MonObjs} : $M \to L$ be a function
   \STATE \textbf{let} \textit{ObjsSym} be a binary relation over L and $\Sigma$
   \STATE \textbf{let} $\psi \in \Psi$ be a finite sequence of method structures
   \STATE \textbf{let} $\eta$ be the data structure holding the program execution contexts
   %\STATE \textbf{let} \textit{getFreq} : A function that takes $\psi \in \Psi$ and $\eta$  and returns \textit{ct} which is the number of times the context was seen
   %\STATE \textbf{let} \textit{incFreq} : A routine that takes $\psi \in \Psi$ and $\eta$  and $n \in \mathcal{N}$ and sets $\psi.count$ 
   
    \IF{$b \in \Sigma_{c}$}
        \STATE $\psi$  $\leftarrow$ getExecutionContextInfo()
        \IF{isMatch($\psi$, $\eta$) = TRUE}
           \STATE $\theta$ $\leftarrow$ threshold($\psi$, $\eta$)
        \ELSE
        	   \STATE $\theta \leftarrow 1$
        \ENDIF
        \STATE updateExecutionContextInfo($\psi$, $\eta$)
        \IF{Rand() $\leq \theta$}  
            	\STATE $m$ $\leftarrow$ $A$.nextMonitor($o$)
        		\FOR{$l' \subseteq$ MonObjs($m$)} 
            	%\IF{$\exists \sigma \in \Sigma : (l', \sigma) \in ObjsSym $}
		\STATE ObjsMons($l'$) $\leftarrow  ObjsMons(l') / \{m\}$
            	%\ENDIF   
        		\ENDFOR
        		\FOR{$l' \subseteq l $} 
            		%\IF{$\exists \sigma \in \Sigma : (l', \sigma) \in ObjsSym $}
			\IF{ObjsMons($l'$).size() = MAX\_MON}
				 \STATE $m' \leftarrow$ ObjsMons($l'$).first();
			  	\FOR{$l'' \subseteq$ MonObjs($m'$)} 
					\STATE ObjsMons($l''$) $\leftarrow$  ObjsMons($l''$) / \{$m'$\}
        				\ENDFOR
			\ENDIF
                		 \STATE ObjsMons($l'$) $\leftarrow$  ObjsMons($l'$) $\cup$ \{$m$\}
            		%\ENDIF   
        		\ENDFOR
        \ENDIF
 \ENDIF
 \FOR{$m$ $\in$ ObjsMons($l$)}
     \STATE $m$.cur $\leftarrow$ $\delta$($m$.cur,$b$)
     \IF{$m$.cur = err}
        \STATE report$\hspace{5pt}\textbf{error}$
     \ENDIF
 \ENDFOR
 
\end{algorithmic}

\end{algorithm}
\label{algo:monitoring}



\subsection{Monitoring Algorithm}
\label{subsec:algo}

Algorithm~1 depicts the steps that implement our monitoring scheme. Lines 9--32 describe the operations that are performed when a creation event is encountered and a new monitor may need to be allocated. Line 10 checks the program execution context. If the execution context is already seen which is checked at  line 11, then a \textit{threshold} value is generated based on the number of times the context is seen in the past; else if the context is unseen, the threshold is assigned the highest possible value which is 1. In either case Line 16 updates the execution context history. In our implementation it involves either adding new branches in the context tree if the context was unseen or only incrementing the frequency count filed of the leaf node corresponding to the known context.

Lines 17--31 describe the steps when the threshold value is found to be large enough to justify allocation of monitor which is checked by the condition at line 17. As a result, a new monitor from the global circular array is allocated. Lines 19--21 describe the steps to reclaim the monitor in case it is previously assigned. This step ensures that all previous bindings are removed and the monitor is ready for the new assignment.

Lines 23--28 describe the steps that ensure that the local monitor pool limit is not reached. In case it is, the oldest monitor in the pool is reclaimed first by removing it from all the lists of associated object maps before the new monitor is added in the lists as shown by line 29.

Finally, as shown in lines 33--38, the relevant monitors are retrieved and their states are updated. In case any of the state is the \textit{error} state, then the error is reported. This final step is similar to the conventional monitoring, except that no monitor will be tracked if the system does not see any monitor allocated.


\subsubsection{Memory-Efficiency and Time-Determinism}
\label{subsubsec:efficiencyanddeterminism}

The algorithm preallocates monitors from a pool of constant size, and then if required, the monitors are reused. In our study we varied the pool size from 100 to 100k monitors. This results in reducing the number of required monitors considerably especially for the challenging program and property combinations in which millions of monitors get generated. As our study indicates, our approach may result in a dramatic saving in the memory requirements of a software.

It is easy to see that the algorithm has worst-case time bounds for all of its steps making the algorithm time-deterministic. Fetching current execution context in the form of a call stack is an expensive but still a constant time operation. The execution context tree has a bounded depth which we have limited to three in our prototype implementation. Hence performing read or write operations on it as in lines 11 and 16 are time-bound operations.

Depending on the map implementations, all the map operations such as the ones on lines 20, 26, and 29 are constant-time operations. The \textit{for} loops on lines 19, 22 and 25 iterate only a small finite number of times depending on the number of objects involved in the event which is typically either one or two. The \textit{for} loop on line 33 executes at most \texttt{MAX\_MON} times since that is the limit on the size of local monitor pools associated with objects.

\subsubsection{Soundness, Memory, and Determinism: Tradeoffs}
\label{subsubsec:tradeoff}

There is unfortunately a tradeoff between soundness and memory and we need to choose one at the cost of other. However, runtime monitoring is inherently unsound \cite{}. It can only report what it has seen. This means program errors may not be reported if the paths that encounter them are not executed. We stretch this limitation a little bit further to achieve substantial benefits in terms of memory savings. Our technique should enable developers to use runtime monitoring in the resource-constrained system settings where the previous techniques for finite-state monitoring might have been seen infeasible.

Similar tradeoff exists between soundness and determinism in terms of time, but we believe that binding worst-case execution times has its own rewards especially in the domain of real-time systems. Hence, our approach tries to use smart heuristics based on our observations and experience so as to reduce false negatives and improve the soundness of the system. As indicated by the study presented in Section~\ref{sec:study}, our approach can save considerable amount memory and can also make monitoring time-determinitic. Moreover, it does not compromise with its \textit{completeness} by ensuring that no false positives are produced.

\ignore{
\section{Simple Code and Property Example:}
While JavaMOP represents state of art runtime monitoring, our goal is to limit the number of monitors associated with a set of objects. It typically models sequencing properties as finite state automata (FSA) and check whether a program satisfies them during runtime.  JavaMOP uses object based monitoring in which a monitor for a FSA property is a relation between a set of related objects created during program execution and a state of the FSA.  The execution of sequence of program statements (under test), related to both property and the set of related objects, corresponds to the sequence of FSA transitions.\\
\begin{figure}[h]
\centering
  \includegraphics[scale=0.6]{./images/Unsafe.jpg}
  \caption[UnsafeIterator Property FSA]{UnSafeIterator Property.}
  \label{fig:typestateProperty FSA}
\end{figure}
\\Let us assume we are monitoring UnsafeIterator property that says not to modify an object of class Collection while iterating over this collection at the same time. In this case, it would be unclear whether the program should iterate over the original contents or the modified contents of the collection. While iterating over a collection, the runtime monitoring ensures that the program behaviour is well-defined. In other words, the monitor checks that there should not be a call to any method that is updating the collection after an iterator is created and before an element is accessed by a call to method next. The regular expression for the property is (create ; next* ; update*) and Fig 3.1 shows the finite state automata for UnsafeIterator property.\\\\
Below shows a code snippet for a test program. The relevant statements for the above property are calls to methods iterator(), add(), and next() on appropriate receiver objects. The relevant statements in the fragment of code being monitored are instrumented with extra code i.e., the OBSERVE... calls to perform monitoring.\\\\
When the call to method iterator() is encountered, a monitoring event create is generated. The instrumented code maintains a map, keyed by set of related objects that have been involved in previous events. The values corresponding to the keys are the set of monitors that are associated with the objects. On handling the create event, if it is found that no monitors are associated with the collection \textit{m1} and iterator \textit{itr}, our approach will list the history of method calls on the object \textit{itr} (bar(a1,a2), third(a1,a2), second(a1,a2), first(a1,a2), main() from our example). It will analyse if there exists same method calls of previously monitored objects and will sample the object for monitoring if match is not found and it will store the history of method calls for this object. However, on finding a match, the object may or may not be considered for monitoring depending upon how frequently it has been monitored previously.
\begin{center}
\textbf{A code snippet}
\begin{lstlisting}

void main(String[] args) {
	  ArrayList a1 = new ArrayList();
     al.add("C");
	  al.add("A");
	  al.add("B"); 
	  ArrayList a2 = new ArrayList ();
	  first(a1,a2);
	}
void first(ArrayList a1, ArrayList a2) { 
	  second(a1,a2); 
	} 
void second(ArrayList a1, ArrayList a2) {
	  third(a1,a2); 
	} 
void third(ArrayList a1, ArrayList a2) {
	  bar(a1,a2); 
	}
void bar(ArrayList m1,Arraylist m2) {
	  Iterator itr = m1.iterator();
		   OBSERVE.create(m1,itr);
	  m2.add(?��X?��);
		   OBSERVE.update(m2);
     while(itr.hasNext()) {
	    Object element = itr.next();
		   OBSERVE.next(itr);
	  }
	}
    
\end{lstlisting}
\end{center}


Whenever an object is sampled for monitoring, the total number of monitors analysing the property are checked. If the number is below the specified limit, a new monitor is created and references to it are associated with the new keys \textit{m1} and \textit{itr} and on the other hand, if the number exceeds the limit, the previously generated monitor is reassigned for the current objects by changing the references. Thus, a limited set of monitors is used for analysing the property violations as done in runtime monitoring. 
}














\ignore{
\begin{algorithm}[h]
                      % enter the algorithm environment
\caption[Algorithm]{Monitoring Algorithm. $\phi$ = (Q,$\Sigma$,$\delta$,$q_{0}$,F), e=(l,b) where e is an event and l $\in \Sigma$ and b $\in O$ be the set of receiver objects}          % give the algorithm a caption
\label{alg1}                           % and a label for \ref{} commands later in the document
\begin{algorithmic}[1]                  
   %\STATE \textbf{let} \textit{O} be the set objects that receive events
   \STATE \textbf{let} $\Sigma_{c} \in \Sigma$ be the set of creation symbols
   \STATE \textbf{let} \textit{prob} determines the generation of monitor (Initializing it to 1).
   \STATE \textbf{let} \textit{MA} be the array of monitors
   \STATE \textbf{let} \textit{ObjMonsMap} : \textit{O} $\to$ \textit{MS} be a map
   \STATE \textbf{let} \textit{ObjsSym} be a binary relation over L and $\Sigma$
   \STATE \textbf{let} \textit{SE} be a finite sequence of method frames
   
    \IF{$b \in \Sigma_{c}  \lor \textit{ObjMonsMap(o)} = null$}
        \STATE $SE  \leftarrow retrieveMethodCalls(o)$
        \FOR{$SE' \subseteq SE$} 
           \STATE $match  \leftarrow
           isMatched(SE')$
        \ENDFOR
        \IF{$(match)$}
           \STATE $prob \leftarrow P(countOfMatches)$
        \ENDIF   
        \IF{$ ((\exists(rand)\subset RNG : rand < prob))$}  
            \IF{$sizeOf (MA) < limit$} 
                \STATE $ m \leftarrow new monitor(o),MA \leftarrow m $
            \ELSE
                \STATE $m \leftarrow retrieve(MA[index]) $
            \ENDIF   
            \STATE $monitoringFlag \leftarrow  true$    
        \ELSE
            \STATE $monitoringFlag \leftarrow  false$ 
        \ENDIF
        \FOR{$l' \subseteq l $} 
            \IF{$\exists \sigma \in \Sigma : (l', \sigma) \in ObjsSym $}
                \IF{$(monitoringFlag) $} 
                   \STATE $ObjsMons(l') \leftarrow  ObjsMons(l') \cup \{m\}$
                \ENDIF
            \ENDIF   
        \ENDFOR
 \ENDIF
 \FOR{$m \in MA$}
     \STATE $m.cur \leftarrow \delta(m.cur,b)$
     \IF{$(m.cur = err)$}
        \STATE report$\hspace{5pt}\textbf{error}$
     \ENDIF
 \ENDFOR
\end{algorithmic}

\end{algorithm}
}

\ignore{Runtime Monitoring is complete but fundamentally unsound since it cannot see anything beyond the current execution. In our approach, we sacrifice on the soundness to achieve memory efficiency and determinism in terms of time.	

The aim of our research is to develop optimization techniques without compromising heavily with the error reporting. We have developed techniques that identify objects in a program for which new monitors will be generated only if they satisfy the required probability related to the program execution context. We are guided by the heuristics that considers the history of method calls that have been invoked on the current object being monitored over the course of object?��s lifetime, from allocation to collection, as the decision factor to determine if the object has been previously monitored.\\\\ 
In object based monitoring, for every object that invokes a property for monitoring, an instance of monitor is created and all subsequent calls on those objects generate events[14]. It may be the case that an object A following a particular sequence of method calls, reaches a method where the typestate property of the current object is monitored for violation and there is another object B which follows the same method call trace as A and reaches the same method. Here, the conventional monitoring approach will create again a new monitor for the second method invocation to check for property violation although the objects of this method have been previously monitored. Thus, in our approach, we are sampling the objects that are to be monitored. The new monitor instances are generated for sampled objects and the objects are sampled by listing the history of method calls invoked on the objects.\\
Our approach is built on dynamic typestate analysis and limits the number of monitors that are associated with events related to a set of objects. Moreover, the total number of monitors generated for the typestate property checking is bounded explicitly in our work. A monitor pool is maintained and once the number of monitors generated reaches the limit specified, the already generated monitors are reassigned for checking the violation of the properties. This helps in utilizing the memory efficiently and reduces memory overhead.\\\\
For a multi-object property, those that involve more than one object such as \textit{UnSafeIterator} property, one \textit{Collection} object has several monitors associated with it at a time. This means that a single method call on that object can result in several monitor
update operations. The number of associated monitors can grow uncontrollably and it becomes difficult to keep a track of every monitor for that event, hence the whole operation of handling events may become non-deterministic in terms of execution time. By limiting the number of monitors associated, the events become bounded and  hence, our approach makes monitoring deterministic in terms of time. We provide worst-case bounds for the execution times of handling events.}
