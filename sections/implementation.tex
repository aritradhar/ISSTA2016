\section{Implemetation and Artifacts}
\label{sec:implementation}

We have developed a semi-automatic prototype


%The aspects are then woven using ajc 1.7 compiler, and 
%the resulting Java program is executed on HPC server running Cent OS 6.5 and
% JVM 1.7.0. 

\subsection{DaCapo instrumentation} 
\label{subsec:dacapoInstr}

We have used \texttt{Soot}~\cite{soot} to
instrument Dacapo benchmarks. \code{getStackTrace()} method can be used to
fetch current stack trace but it introduces overhead. To mitigate this, we have
instrumented each of the methods of DaCapo benchmark with a static
\code{integer} field which is populated with an unique method id represented by
a $16$ bit \code{integer}. We have also simulate the stack trace by using a
circular array which contains these method id's. The circular array is
implemented by a $64$ bit \code{long}. A pseudo code of the method stack is
described in Code~\ref{snippet:methodTrace}.

\lstset{escapeinside={/*@}{@*/}, language=Java , caption=\bf Method trace
implementation., label=snippet:methodTrace} \begin{figure}[t]
\begin{lstlisting}
long trace; //method stack trace	
int counter = 0;
void methodTrace(int methodID) {
 switch (counter) {
  case 0: id <<= 32; break;
  case 1: id <<= 16; break
  case 2: id <<= 0; break;
  default: break;
 }
 trace |= id;
 trace &= 0xffffffffffffL;
 counter = (counter + 1) % traceLength;
}
\end{lstlisting}
\end{figure}

\subsection{Aspect Generation and Optimizations}
\label{subsec:aspectGen}

Our implementation is semi-automatic and is based on the refinement of the 
aspects generated by JavaMOP 2.3. The reason for choosing an older version of 
JavaMOP was that it is simple to understand and it performs less optimizations. 
Hence, the chance of interfering the existing optimizations with our approach is 
less. This would not have been possible if we did not have a good understanding 
of the aspects consumed by our prototype implementation. Moreover, we compare 
the results of our work with JavaMOP 2.3 as well as JavaMOP 3.0 which is a much 
newer version of JavaMOP. 


\subsection{Context Matching}
\label{subsec:contextMatch}
The time and memory overheads have been reported after using -converge option 
provide by the DaCapo benchmark suite.